\documentclass{tufte-book}

\hypersetup{colorlinks}% uncomment this line if you prefer colored hyperlinks (e.g., for onscreen viewing)

%%
% Book metadata
\title{Escape the Dungeon\thanks{Lumpley, Avery and the RPGtalk slackers }}
\author[Cromlyn Games]{Cromlyn Games}
\publisher{Publisher of This Book}

%%
% If they're installed, use Bergamo and Chantilly from www.fontsite.com.
% They're clones of Bembo and Gill Sans, respectively.
%\IfFileExists{bergamo.sty}{\usepackage[osf]{bergamo}}{}% Bembo
%\IfFileExists{chantill.sty}{\usepackage{chantill}}{}% Gill Sans

%\usepackage{microtype}

%%
% Just some sample text
\usepackage{lipsum}

%for fancy lists
\usepackage{tikz}
\usetikzlibrary{shadows}
\newcommand{\mylist}{\tikz[overlay]\draw(-.2,-.2)--(-.2,.4) [path fading=east](-.2,.15)--(.1,.15);} %adds the |- shape to the start of each list item
\newcommand{\mylistend}{\tikz[overlay]\draw(-.2,.15)--(-.2,.4) [path fading=east](-.2,.15)--(.1,.15);} %adds the |- shape to the start of each list item
\newcommand{\myitem}{\item[\mylist]} %defines the scope of the mylist command to be 2nd level sublists
\newcommand{\myitemend}{\item[\mylistend]} %defines the scope of the mylist command to be 2nd level sublists

%%
% For nicely typeset tabular material
\usepackage{booktabs}

%%
% For graphics / images
\usepackage{graphicx}
\setkeys{Gin}{width=\linewidth,totalheight=\textheight,keepaspectratio}
\graphicspath{{graphics/}}

% The fancyvrb package lets us customize the formatting of verbatim
% environments.  We use a slightly smaller font.
\usepackage{fancyvrb}
\fvset{fontsize=\normalsize}

%%
% Prints argument within hanging parentheses (i.e., parentheses that take
% up no horizontal space).  Useful in tabular environments.
\newcommand{\hangp}[1]{\makebox[0pt][r]{(}#1\makebox[0pt][l]{)}}

%%
% Prints an asterisk that takes up no horizontal space.
% Useful in tabular environments.
\newcommand{\hangstar}{\makebox[0pt][l]{*}}

%%
% Prints a trailing space in a smart way.
\usepackage{xspace}

%%
% Some shortcuts for Tufte's book titles.  The lowercase commands will
% produce the initials of the book title in italics.  The all-caps commands
% will print out the full title of the book in italics.
\newcommand{\vdqi}{\textit{VDQI}\xspace}
\newcommand{\ei}{\textit{EI}\xspace}
\newcommand{\ve}{\textit{VE}\xspace}
\newcommand{\be}{\textit{BE}\xspace}
\newcommand{\VDQI}{\textit{The Visual Display of Quantitative Information}\xspace}
\newcommand{\EI}{\textit{Envisioning Information}\xspace}
\newcommand{\VE}{\textit{Visual Explanations}\xspace}
\newcommand{\BE}{\textit{Beautiful Evidence}\xspace}

\newcommand{\TL}{Tufte-\LaTeX\xspace}

% Prints the month name (e.g., January) and the year (e.g., 2008)
\newcommand{\monthyear}{%
  \ifcase\month\or January\or February\or March\or April\or May\or June\or
  July\or August\or September\or October\or November\or
  December\fi\space\number\year
}


% Prints an epigraph and speaker in sans serif, all-caps type.
\newcommand{\openepigraph}[2]{%
  %\sffamily\fontsize{14}{16}\selectfont
  \begin{fullwidth}
  \sffamily\large
  \begin{doublespace}
  \noindent\allcaps{#1}\\% epigraph
  \noindent\allcaps{#2}% author
  \end{doublespace}
  \end{fullwidth}
}

% Inserts a blank page
\newcommand{\blankpage}{\newpage\hbox{}\thispagestyle{empty}\newpage}

\usepackage{units}

% Typesets the font size, leading, and measure in the form of 10/12x26 pc.
\newcommand{\measure}[3]{#1/#2$\times$\unit[#3]{pc}}

% Macros for typesetting the documentation
\newcommand{\hlred}[1]{\textcolor{Maroon}{#1}}% prints in red
\newcommand{\hangleft}[1]{\makebox[0pt][r]{#1}}
\newcommand{\hairsp}{\hspace{1pt}}% hair space
\newcommand{\hquad}{\hskip0.5em\relax}% half quad space
\newcommand{\TODO}{\textcolor{red}{\bf TODO!}\xspace}
\newcommand{\na}{\quad--}% used in tables for N/A cells
\providecommand{\XeLaTeX}{X\lower.5ex\hbox{\kern-0.15em\reflectbox{E}}\kern-0.1em\LaTeX}
\newcommand{\tXeLaTeX}{\XeLaTeX\index{XeLaTeX@\protect\XeLaTeX}}
% \index{\texttt{\textbackslash xyz}@\hangleft{\texttt{\textbackslash}}\texttt{xyz}}
\newcommand{\tuftebs}{\symbol{'134}}% a backslash in tt type in OT1/T1
\newcommand{\doccmdnoindex}[2][]{\texttt{\tuftebs#2}}% command name -- adds backslash automatically (and doesn't add cmd to the index)
\newcommand{\doccmddef}[2][]{%
  \hlred{\texttt{\tuftebs#2}}\label{cmd:#2}%
  \ifthenelse{\isempty{#1}}%
    {% add the command to the index
      \index{#2 command@\protect\hangleft{\texttt{\tuftebs}}\texttt{#2}}% command name
    }%
    {% add the command and package to the index
      \index{#2 command@\protect\hangleft{\texttt{\tuftebs}}\texttt{#2} (\texttt{#1} package)}% command name
      \index{#1 package@\texttt{#1} package}\index{packages!#1@\texttt{#1}}% package name
    }%
}% command name -- adds backslash automatically
\newcommand{\doccmd}[2][]{%
  \texttt{\tuftebs#2}%
  \ifthenelse{\isempty{#1}}%
    {% add the command to the index
      \index{#2 command@\protect\hangleft{\texttt{\tuftebs}}\texttt{#2}}% command name
    }%
    {% add the command and package to the index
      \index{#2 command@\protect\hangleft{\texttt{\tuftebs}}\texttt{#2} (\texttt{#1} package)}% command name
      \index{#1 package@\texttt{#1} package}\index{packages!#1@\texttt{#1}}% package name
    }%
}% command name -- adds backslash automatically
\newcommand{\docopt}[1]{\ensuremath{\langle}\textrm{\textit{#1}}\ensuremath{\rangle}}% optional command argument
\newcommand{\docarg}[1]{\textrm{\textit{#1}}}% (required) command argument
\newenvironment{docspec}{\begin{quotation}\ttfamily\parskip0pt\parindent0pt\ignorespaces}{\end{quotation}}% command specification environment
\newcommand{\docenv}[1]{\texttt{#1}\index{#1 environment@\texttt{#1} environment}\index{environments!#1@\texttt{#1}}}% environment name
\newcommand{\docenvdef}[1]{\hlred{\texttt{#1}}\label{env:#1}\index{#1 environment@\texttt{#1} environment}\index{environments!#1@\texttt{#1}}}% environment name
\newcommand{\docpkg}[1]{\texttt{#1}\index{#1 package@\texttt{#1} package}\index{packages!#1@\texttt{#1}}}% package name
\newcommand{\doccls}[1]{\texttt{#1}}% document class name
\newcommand{\docclsopt}[1]{\texttt{#1}\index{#1 class option@\texttt{#1} class option}\index{class options!#1@\texttt{#1}}}% document class option name
\newcommand{\docclsoptdef}[1]{\hlred{\texttt{#1}}\label{clsopt:#1}\index{#1 class option@\texttt{#1} class option}\index{class options!#1@\texttt{#1}}}% document class option name defined
\newcommand{\docmsg}[2]{\bigskip\begin{fullwidth}\noindent\ttfamily#1\end{fullwidth}\medskip\par\noindent#2}
\newcommand{\docfilehook}[2]{\texttt{#1}\index{file hooks!#2}\index{#1@\texttt{#1}}}
\newcommand{\doccounter}[1]{\texttt{#1}\index{#1 counter@\texttt{#1} counter}}

% Generates the index
\usepackage{makeidx}
\makeindex

\begin{document}

% Front matter
\frontmatter

% r.1 blank page
\blankpage

% v.2 epigraphs
\newpage\thispagestyle{empty}
\openepigraph{%
The public is more familiar with bad design than good design.
It is, in effect, conditioned to prefer bad design, 
because that is what it lives with. 
The new becomes threatening, the old reassuring.
}{Paul Rand%, {\itshape Design, Form, and Chaos}
}
\vfill

\openepigraph{%
Acknowledge up front that the PCs are going to win, and never sweat it. Then use the dice to escalate, escalate, escalate. We all know the PCs are going to win. What will it cost them?
}{Lumpley}
\vfill

\openepigraph{%
A designer knows that he has achieved perfection 
not when there is nothing left to add, 
but when there is nothing left to take away.
}{Antoine de Saint-Exup\'{e}ry}
\vfill
\openepigraph{%
Divandra and I have now returned to full health, and it is time to go on: hacking and slashing, looting and robbing, opening every box and barrel in the hope that we may unearth a clue as to what this is all about. God send that it is not a vain hope.
}{ Ernest Adams}
\vfill




% r.3 full title page
\maketitle


% v.4 copyright page
\newpage
\begin{fullwidth}
~\vfill
\thispagestyle{empty}
\setlength{\parindent}{0pt}
\setlength{\parskip}{\baselineskip}
Copyright \copyright\ \the\year\ \thanklessauthor

\par\smallcaps{Published by \thanklesspublisher}

\par\smallcaps{tufte-latex.github.io/tufte-latex/}

\par Licensed under the Apache License, Version 2.0 (the ``License''); you may not
use this file except in compliance with the License. You may obtain a copy
of the License at \url{http://www.apache.org/licenses/LICENSE-2.0}. Unless
required by applicable law or agreed to in writing, software distributed
under the License is distributed on an \smallcaps{``AS IS'' BASIS, WITHOUT
WARRANTIES OR CONDITIONS OF ANY KIND}, either express or implied. See the
License for the specific language governing permissions and limitations
under the License.\index{license}

\par\textit{First printing, \monthyear}
\end{fullwidth}

% r.5 contents
\tableofcontents

\listoffigures

\listoftables

% r.7 dedication
\cleardoublepage
~\vfill
\begin{doublespace}
\noindent\fontsize{18}{22}\selectfont\itshape
\nohyphenation
Dedicated to those who appreciate \LaTeX{} 
and the work of \mbox{Edward R.~Tufte} 
and \mbox{Donald E.~Knuth}.

And to the horde of goblins who died defending them.
\end{doublespace}
\vfill
\vfill


% r.9 introduction
\cleardoublepage
\chapter*{Introduction}


In this game you are in a typical fantasy dungeon. The overlord is dead \footnote {citation needed}. You were one of his underlings. You were a monster. You still are, but for the first time in a long time, you're able to ask "Why?" So can all the other underlings.
\smallcaps{What do you do?}

We're going to start with the base outline of what the GM should be doing, then the core of what the player needs to know, then some playbooks of different underlings and then all the stuff the GM needs to show you an awesome time.

%%
% Start the main matter (normal chapters)
\mainmatter


\chapter{GM Agenda and Principles}
\label{ch:agenda}


\newthought{The Game Master} or GM is the referee. Master of Ceremonies in Apocalaypse World, Dungeon Master in old-school hack and slash grid crawlers, they provide the stage, the setting, most of the background characters and the physics engine. They are here to make the game fun for you. If they're not doing that, either they're not running the game right, I've written it wrong, or you should both try a different game that suits what you are seeking.\footnote{this is a serious point really, it's worth taking the time at the start to  discuss palette and previous games or stories you've enjoyed. This will be discussed more in the GM section}


\bigskip
\section{Agenda}\label{sec:Agenda}\index{agenda}
\begin{enumerate}
   %\begin{itemize}
    \item Agenda
	\begin{itemize}
	\myitem Play to find out what happens
	\myitem Make the players'  character's lives not boring
	\myitemend Subvert that which is taken from granted
	\end{itemize}
   %\end{itemize}
\end{enumerate}
The Agenda is the key thing. If you, as GM, are in doubt what to do, choose the thing that follows the agenda. It's not a bad guide for players either.

\bigskip
\section{Principles}\label{sec:Principles}\index{principles}
\begin{enumerate}
   %\begin{itemize}
	\item Principles
	\begin{itemize}
	\myitem don't waste your players' time
	\myitem the dungeon pressures to conform
	\myitem take tropes to their logical, nonsensical extreme
	\myitem sprinkle details of everyday fantasy everywhere 
	\myitem make the dungeon seem fantastically real
	\myitem name everyone, make everyone rational within their role
	\myitem build a bigger dungeon through play, not plot
	\myitem create interesting dilemmas not interesting traps
	\myitem address yourself to the characters not players
	\myitem make your move, but never speak its name
	\myitem ask loaded questions and build on the answers 
	\myitemend sometimes, reflect a question back upon the players 
	\end{itemize}
%\end{itemize}
\end{enumerate}
\bigskip
The Principles are what the GM should be doing if they have their mouth open. If she's chewing pizza instead, call her out on it. Exactly how they translate to the fiction will vary on the tone of the game. These rules are supposed to support a game with a strong undercurrent of pathos with the silliness. The situation is ridiculous, because a lot of fantasy is ridiculous, but a lot of the player characters should be sympathetic or relatable as they struggle in this crazy situation.\footnote{A bit like most sitcoms}  Fundamentally, they are outside of the mainstream surface society and trying to retain a sense of identity when it'd be much easier to slide into institutionalisation. The Dungeon only wants to help, to protect them from the weird outside world where there's no roof and much less glowy rocks. The GM represents the dungeon.


\chapter{Player's core}

You play as an underling, someone or something recently awoken from your role in the broader dungeon. The dungeon ain't to happy about that, by the way. You'll have a playbook with some moves unique to you, some stats to help define how good you are at different things, some \smallcaps{demons} that represent the things you fear and probably some other stuff like a tribe or some mates or a hoard or a soul-sucking artifact of glowy evilness. You also have access to the basic moves. Everyone has them, and they should be pretty useful.

There are six main stats in this game.
\begin{enumerate}
%\begin{itemize}
\item \smallcaps{WIZARD} - how good you are brainpower, thinking or arcane magic stuff
\item \smallcaps{THIEF} - how good you are with cunning and delicate or precise skill
\item \smallcaps{FIGHTER} - how good you are with violence or raw strength
\item \smallcaps{BARD} - how good you are with charm and social connections
\item \smallcaps{Identity} - how good you are at remembering who you actually are beneath the stereotypes.
\item \smallcaps{Health} - how much more damage you can take, for now
%\end{itemize}
\end{enumerate}

They'll get set in your playbook. Different underlings have different stats. They may change over play too. Identity certainly will. Unless the move specifically says so, none of those five stats can go above +3 (or below -2, if you are a masochist). A move that uses a stat will state something like \smallcaps{roll + stat name}. That means roll two  normal six-sided dice and add the total to your stat. If you have \smallcaps{wizard} of -1, and you roll a 2 and a 5 the total is 6. Most rules use the format of get 6 or below total and you've messed up, get 7-9 and you succeed, but at a cost. Rolling 10+ is a success, sometimes with a bonus. Try to do that.
Sometime you'll see things like a +1 forward. That means you get to add +1 to your next roll. In the case above, that'd be enough to turn the 6 into 7 and the dangerous failure into a dangerous success.

\marginnote{In play, don't make moves. Do stuff, and keep doing stuff until the GM calls for a move. Dice should only hit the table when the stakes are interesting and the outcome uncertain. You don't need a \smallcaps{roll+thief} to use a doorhandle. You don't even need it to pick a lock when you've got all afternoon and someone making you cups of tea. When you're trapped in a corridor with fire elementals drifting towards you from both sides and your buddy is bleeding out all over your feet, yeah, then you need to roll.} 


\smallcaps{health} is the sixth stat and typically starts at nine and fluctuates wildly if you are playing hard enough. You heal up 3 health in a \smallcaps{long rest} - which means a night\footnote{how do underlings in your dungeon keep track of time?} in the fiction, or probably the gap between sessions in real life. Don't hoard your health, we'll cover running out under basic moves.

During play, or possibly starting out, you'll pick up items of various use. The 'item' is the basic unit of currency in the dungeon, those pesky heroes keep dropping them when they die and there's not a lot of other use for half the stuff it seems. The playbook is also a good place to record favours, debts and stuff you want to go back to later. It's also the place to record hobbies and Xp. 

\smallcaps{Hobbies}, in this game, are important. They keep you grounded and bolster your sense of identity. Every time you take the time to act out your hobby, you mark a little tick next to it. Every three ticks means +1 to your identity stat.  

\smallcaps{Xp} stands for Experience and represents  how far down the path to the extreme expression of your stereotype you've walked, lurched or slithered. Collecting \smallcaps{Xp} will unlock more moves, stat bonuses, moves from other playbooks or more major character development. These advances will be shown in the playboook.

\smallcaps{objectives vs damage}. In the basic moves below, commonly you will be told you've succeeded at your objective OR you deal damage. There are two intentions here. The first is to allow a move to be used for different problems where damage is not appropriate: "I want to climb out of the pit", "I want to pick the lock", "I want to study the runes, see if I can find a clue to the ritual." The second intention is to clarify how much damage you do if your objective is something like "I move to stab her in the back," "I exhale and shoot an arrow at that gap in his armour.". Basically, you can't state your objective is to 'behead the world-turtle' and succeed when Great A'Tuin still has a few thousand hitpoints left.\footnote{And the corollary is that the GM should note that relying on hitpoints is bad challenge design. You should be building a jungle-gym not a treadmill. There's an excellent essay for Dungeon World called "The 16Hp Dragon" that I recommend.} As a rule, the \smallcaps{fighter} should be able to deal the most damage to a single target, the \smallcaps{thief} can spread it around but might go through a lot of vials of poison or throwing knives and the \smallcaps{Wizard} is an unstable powderkeg of potential.
\bigskip

\section{WIZARD: basic move}
When you try to solve a problem with raw brainpower, knowledge or magic \smallcaps{roll+wizard}.
\begin{enumerate}
%\begin{itemize}
\item On a 7+ you succeed at your objective OR you cast a damaging spell of raw magic.\footnote{This represents basic, almost instinctual attacks. Detailed wizard spells are in the playbooks. The difference is like hitting someone with a ripped open electricity wire and constructing a freezer, or a drill or a light bulb that can take the electricity and make something new with it.}
\item On a 10+ you succeed at your objective. Choose two:
	\begin{itemize}
	\myitem Deal one damage to someone
	\myitem Deal one damage to all in the room
	\myitem Expose a weakness, flaw or demon
	\myitem You give good advice. You give an ally +1 forward.
	\myitemend Learn three words about the target
	\end{itemize}
\item On a 7-9 you succeed at your objective AND deal one damage to someone
\item On a 6- you deal one damage to yourself or an ally
%\end{itemize}
\end{enumerate}
\bigskip




\section{THIEF: basic move}
When you try to solve a problem with cunning, treachery or precise and delicate application of skill \smallcaps{roll+ thief}
\footnote{This is the stat to roll for stealth, or dodging arrows too. When it comes to lies and blather, there is some overlap with Bard moves. The GM is encouraged to be relaxed about this, but keep an eye on the consequences of a miss. }
\begin{enumerate}
%\begin{itemize}
\item on a 7+ you succeed at your objective OR deal one damage\footnote{if your objective was to deal damage, then deal damage.}
\item on a 10+ choose three from below
\item on a 7-9 choose one:
	\begin{itemize}
	\myitem Deal one damage to some other target
	\myitem You do it quickly
	\myitem You can get away cleanly
	\myitem It can't be traced to you
	\myitemend You don't use up an item
	\item Now the GM chooses a remaining option to move against. 
	\end{itemize}
\item On a 6- the GM may choose two options.
%\end{itemize}
\end{enumerate}
\bigskip

\section{FIGHTER: basic move}
When you try to solve a problem with violence, strength or sheer athleticism \smallcaps{Roll + Fighter}:
\footnote{This is the stat to roll for breaking down doors, taking blows on a shield or arm-wrestling in the ork canteen. There is some blurriness with \smallcaps{thief}. Stabbing someone in a fight is a fighter move, but stabbing them in the back is a thief. Shooting a charging hero is direct violence and typically therefore \smallcaps{fighter}, but trying to pull off a trick shot that pins their dagger hand to the wall might be \smallcaps{Thief}. The GM is encouraged to be relaxed about this, but encourage thought about the consequence of a miss. If players are pushing \smallcaps{thief} to try and avoid blowback damage, the GM should hit them with some very hard direct moves when a golden opportunity arrives. Players. I hope you read this.}
\begin{enumerate}
%\begin{itemize}
\item On a 7+ you succeed at you objective OR deal two damage.
\item On a 10+: that's it, you succeeded. Deal with the consequences.
\item On a 7-9: the player chooses two:
\begin{itemize}
	\myitem Take damage as the situation demands
	\myitem An ally takes damage as established
	\myitem Something in the next room hears you
	\myitem You drop an item in the middle of it
	\myitem You choose to fail your objective to prevent something worse
	\myitemend Click! next move by anyone triggers a trap 
	\end{itemize}
\item On a 6 or below: no success, but choose one from the list anyway
%\end{itemize}
\end{enumerate}

\bigskip

\section{BARD: basic move}
When you try to solve a problem with charm, social connections or distraction, \smallcaps{roll+bard}:
\begin{enumerate}
%\begin{itemize}
\item On a 7+ you succeed at your objective
\item On a 10+: that's it, you succeeded. Deal with the consequences.
\item On a 7-9 choose two:
	\begin{itemize}
	\myitem You distracted an ally too\footnote{be careful with player agency here. It can be be a fun running gag, but make sure the other player is on board. }
	\myitem You allow an ally to choose the second option for you
	\myitem The target becomes obsessed with you
	\myitem You owe someone. 
	\myitem Someone will come after you, later
	\myitem They offer you a further opportunity, with a catch
	\end{itemize}
\item on a 6- the GM chooses another player who chooses one for you.
%\end{itemize}
\end{enumerate}

\bigskip

\section{IDENTITY: basic move}
When you draw on knowledge of the dungeon, or wake after a \smallcaps{long rest}, then say the thing you know because of your role in the dungeon. You get that statement free. Now \smallcaps{roll + identity}
\begin{enumerate}
%\begin{itemize}
	\item On a 10+ Describe a hobby\footnote{Ideally this should be something outside of your stereotypical job. It's something you value and think about in your spare time and makes plans for. 'Three different times' doesn't mean three cups of tea satisfies a 'tea party' hobby for a Brute. It has to cost something}. Add it to your playbook. Gain +1 Identity when you have acted on that hobby three different times.
	\item On a 7-9you keep a sense of identity but the dungeon presses on you. Choose one:
	\begin{itemize}
	\myitem I loved that part of my job
	\myitem I hate that part of my job
	\myitemend I had a fierce rival in my job 
	\end{itemize}
	\item On a 6- you slip back into your role. Mindlessly do what your job demands until obstructed or you fail a roll.
%\end{itemize}
\end{enumerate}
\bigskip

\section{One Foot in the Grave: Basic move}
When your health hits zero choose one:
\begin{enumerate}
%\begin{itemize}
\item Die free.
\item Come Back: The dungeon isn't done with you. Take -1 to your two highest stats, and wake up later with 3Hp and holes in your memory.
\item Undergo a sea change: The dungeon has a new role for you. Switch to a new playbook as the story demands.\footnote{\smallcaps{undead} or \smallcaps{creep} are obvious, but hideous experiments could bring you back as a \smallcaps{Hybrid, Brute, Beast} ect.} Unless the GM feels merciful assume all your items are stolen while your are out. You keep one hobby and one move but switch stats, including identity. 
\item 'He who kills monsters': where it makes sense, you may come back as the NPC who killed you. Choose an appropriate playbook that's not in use and get with the GM.
%\end{itemize}
\end{enumerate}


\section{Help or Interfere}
Most Powered By the Apocalypse games include a move that allows you to modify someone else's roll if you take concrete action to support their action. In this game, this is handled in playbook moves and the \smallcaps{Wizard} move. If it is player character vs player character, you don't both \smallcaps{Roll+Fighter} to see who wins. The first person within the fiction to do something that triggers the move does the move, finishes the move and then the fiction continues.\footnote{Designer's note. I am a bit worried about how this will work out. Every game seems to try something different to Hx, and non seem to have settled on a good solution}



%% --------------------------------------------------------

\chapter{Brute}

Statistics: \smallcaps{Fighter +3}. +1 and -1 to any of the others.
\bigskip
Job(Pick 1): Guard; Gladiator; Miner; Chef
\bigskip
Demon(Pick 3): Weakness; Stupidity; Chains; Accidental Harm; Electricity
\bigskip
Hobbies (fill in during play): 
\bigskip

Start with two moves.

\section{Gut Feeling: Brute Move}
When you witness a successful \smallcaps{fighter} move then you can \smallcaps{roll + wizard}:
\begin{enumerate}
%\begin{itemize}
\item On a 10+ you may ask two of the GM or the player.
\item On a 7-9 you may ask one:
	\begin{itemize}
	\myitem What motivates this fight?
	\myitem What would they kill for?
	\myitem What demon hunts them?
	\myitem What must I beware of?
	\myitemend Who trained them?
	\end{itemize}
\item On a 6- you join the fight instead.
%\end{itemize}
\end{enumerate}

\section{Roaring Charge: Brute Move}
When you charge into another player's fight you may destroy the weapon you are holding to get +1 Fight (to a max of +4) for the scene.

\section{Couple of mates: Brute Move}
You have a couple of stupid and unruly mates. Name them. Name the body part that drives them.
When you want them to do anything other than join a fight\footnote{sometimes including stopping again} then \smallcaps{roll + bard}
\begin{enumerate}
%\begin{itemize}
\item On a 10+ they do it just like you asked.
\item On a 7-9 choose one:
	\begin{itemize}
	\myitem They do it, but really badly
	\myitem They do it, but need bribing
	\myitem They do it, but will take it out on an ally later
	\myitemend They obey like the dungeon demands. Take -1 Identity.
	\end{itemize}
\item On a 6- they do their own thing.
%\end{itemize}
\end{enumerate}

\section{It's a Weapon!: Brute Move}
When you have taken damage and are weaponless, \smallcaps{Roll + fighter}
\begin{enumerate}
\begin{itemize}
\item On a 7+ you find something to use as a weapon
\item On a 10+ choose one:
	\begin{itemize}
	\myitem Improvised shield - you or an ally may ignore the next damage
	\myitem Improvised missile - throw it to deal one damage and take the target out of the fight until you move again.
	\myitem Puny elf - you overpower one enemy and use them as a weapon to hit others.
	\end{itemize}
\item On a 6- you find something, but can't get to it quickly enough
\end{itemize}
\end{enumerate}

\section{Scarred: Brute Move}
When you take damage and reach your last health (1Hp), \smallcaps{mark xp} and resolve any new advances immediately. 

%% ---   ---   ---   ---   ---   

\chapter{Creep} % formerly Shadowkin

%% ---   ---   ---   ---   ---   
\chapter{Undead}

%% ---   ---   ---   ---   ---   
\chapter{Horde}

%% ---   ---   ---   ---   ---   
\chapter{Beast}

%% ---   ---   ---   ---   ---   
\chapter{Hybrid}

%% ---   ---   ---   ---   ---   
\chapter{Construct}

%% ---   ---   ---   ---   ---   
\chapter{Magus} % formerly Wizard

%% ---   ---   ---   ---   ---   
\chapter{Dregs}

%% ---   ---   ---   ---   ---   
\chapter{Plant}

%% ---   ---   ---   ---   ---   
\chapter{Fey}

%% ---   ---   ---   ---   ---   
\chapter{Host}

%% ---   ---   ---   ---   ---   
\chapter{Elemental Ooze}








%%
% The back matter contains appendices, bibliographies, indices, glossaries, etc.







\backmatter

\bibliography{sample-handout}
\bibliographystyle{plainnat}


\printindex

\end{document}


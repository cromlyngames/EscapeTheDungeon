\documentclass{tufte-book}

\hypersetup{colorlinks}% uncomment this line if you prefer colored hyperlinks (e.g., for onscreen viewing)

%%
% Book metadata
\title{Escape the Dungeon\thanks{Lumpley, Avery and the RPGtalk slackers }}
\author[Cromlyn Games]{Cromlyn Games}
\publisher{Publisher of This Book}

%%
% If they're installed, use Bergamo and Chantilly from www.fontsite.com.
% They're clones of Bembo and Gill Sans, respectively.
%\IfFileExists{bergamo.sty}{\usepackage[osf]{bergamo}}{}% Bembo
%\IfFileExists{chantill.sty}{\usepackage{chantill}}{}% Gill Sans

%\usepackage{microtype}

%%
% Just some sample text
\usepackage{lipsum}

%for fancy lists
\usepackage{tikz}
\usetikzlibrary{shadows}
\newcommand{\mylist}{\tikz[overlay]\draw(-.2,-.2)--(-.2,.4) [path fading=east](-.2,.15)--(.1,.15);} %adds the |- shape to the start of each list item
\newcommand{\mylistend}{\tikz[overlay]\draw(-.2,.15)--(-.2,.4) [path fading=east](-.2,.15)--(.1,.15);} %adds the |- shape to the start of each list item
\newcommand{\myitem}{\item[\mylist]} %defines the scope of the mylist command to be 2nd level sublists
\newcommand{\myitemend}{\item[\mylistend]} %defines the scope of the mylist command to be 2nd level sublists
%\renewcommand\labelitemi{---}  % turns bullets into long dashes

%%
% For nicely typeset tabular material
\usepackage{booktabs}

%%
% For graphics / images
\usepackage{graphicx}
\setkeys{Gin}{width=\linewidth,totalheight=\textheight,keepaspectratio}
\graphicspath{{graphics/}}

% The fancyvrb package lets us customize the formatting of verbatim
% environments.  We use a slightly smaller font.
\usepackage{fancyvrb}
\fvset{fontsize=\normalsize}

%%
% Prints argument within hanging parentheses (i.e., parentheses that take
% up no horizontal space).  Useful in tabular environments.
\newcommand{\hangp}[1]{\makebox[0pt][r]{(}#1\makebox[0pt][l]{)}}

%%
% Prints an asterisk that takes up no horizontal space.
% Useful in tabular environments.
\newcommand{\hangstar}{\makebox[0pt][l]{*}}

%%
% Prints a trailing space in a smart way.
\usepackage{xspace}

% Prints the month name (e.g., January) and the year (e.g., 2008)
\newcommand{\monthyear}{%
  \ifcase\month\or January\or February\or March\or April\or May\or June\or
  July\or August\or September\or October\or November\or
  December\fi\space\number\year
}


% Prints an epigraph and speaker in sans serif, all-caps type.
\newcommand{\openepigraph}[2]{%
  %\sffamily\fontsize{14}{16}\selectfont
  \begin{fullwidth}
  \sffamily\large
  \begin{doublespace}
  \noindent\allcaps{#1}\\% epigraph
  \noindent\allcaps{#2}% author
  \end{doublespace}
  \end{fullwidth}
}

% Inserts a blank page
\newcommand{\blankpage}{\newpage\hbox{}\thispagestyle{empty}\newpage}

\usepackage{units}

% Typesets the font size, leading, and measure in the form of 10/12x26 pc.
\newcommand{\measure}[3]{#1/#2$\times$\unit[#3]{pc}}

% Macros for typesetting the documentation

% Generates the index
\usepackage{imakeidx}
%\makeindex[name=moves, title={Index of moves}] % on reflection, not needed. 
\makeindex[name=stuff, title ={Index of elements, items}]
\makeindex % general index for playbooks and stuff

\begin{document}

% Front matter
\frontmatter

% r.1 blank page
\blankpage

% v.2 epigraphs
\newpage\thispagestyle{empty}
\openepigraph{%
The public is more familiar with bad design than good design.
It is, in effect, conditioned to prefer bad design, 
because that is what it lives with. 
The new becomes threatening, the old reassuring.
}{Paul Rand%, {\itshape Design, Form, and Chaos}
}
\vfill

\openepigraph{%
Acknowledge up front that the PCs are going to win, and never sweat it. Then use the dice to escalate, escalate, escalate. We all know the PCs are going to win. What will it cost them?
}{Lumpley}
\vfill

\openepigraph{%
A designer knows that he has achieved perfection 
not when there is nothing left to add, 
but when there is nothing left to take away.
}{Antoine de Saint-Exup\'{e}ry}
\vfill
\openepigraph{%
Divandra and I have now returned to full health, and it is time to go on: hacking and slashing, looting and robbing, opening every box and barrel in the hope that we may unearth a clue as to what this is all about. God send that it is not a vain hope.
}{ Ernest Adams}
\vfill




% r.3 full title page
\maketitle


% v.4 copyright page
\newpage
\begin{fullwidth}
~\vfill
\thispagestyle{empty}
\setlength{\parindent}{0pt}
\setlength{\parskip}{\baselineskip}
Copyright \copyright\ \the\year\ \thanklessauthor

\par\smallcaps{Published by \thanklesspublisher}

\par\smallcaps{tufte-latex.github.io/tufte-latex/}

\par Licensed under the Apache License, Version 2.0 (the ``License''); you may not
use this file except in compliance with the License. You may obtain a copy
of the License at \url{http://www.apache.org/licenses/LICENSE-2.0}. Unless
required by applicable law or agreed to in writing, software distributed
under the License is distributed on an \smallcaps{``AS IS'' BASIS, WITHOUT
WARRANTIES OR CONDITIONS OF ANY KIND}, either eXpress or implied. See the
License for the specific language governing permissions and limitations
under the License.\index{license}

\par\textit{First printing, \monthyear}
\end{fullwidth}

% r.5 contents
\tableofcontents

\listoffigures

\listoftables

% r.7 dedication
\cleardoublepage
~\vfill
\begin{doublespace}
\noindent\fontsize{18}{22}\selectfont\itshape
\nohyphenation
Dedicated to those who appreciate \LaTeX{} 
and the work of \mbox{Edward R.~Tufte} 
and \mbox{Donald E.~Knuth}.

And to the horde of goblins who died defending them.
\end{doublespace}
\vfill
\vfill


% r.9 introduction
\cleardoublepage
\chapter*{Introduction}


In this game you are in a typical fantasy dungeon. The overlord is dead \footnote {citation needed}. You were one of his underlings. You were a monster. You still are, but for the first time in a long time, you're able to ask "Why?" So can all the other underlings.
\smallcaps{What do you do?}

We're going to start with the base outline of what the GM should be doing, then the core of what the player needs to know, then some playbooks of different underlings and then all the stuff the GM needs to show you an awesome time.

%%
% Start the main matter (normal chapters)
\mainmatter


\chapter{GM Agenda and Principles}
\label{ch:agenda}


\newthought{The Game Master} or GM is the referee. Master of Ceremonies in Apocalaypse World, Dungeon Master in old-school hack and slash grid crawlers, they provide the stage, the setting, most of the background characters and the physics engine. They are here to make the game fun for you. If they're not doing that, either they're not running the game right, I've written it wrong, or you should both try a different game that suits what you are seeking.\footnote{this is a serious point really, it's worth taking the time at the start to  discuss palette and previous games or stories you've enjoyed. This will be discussed more in the GM section}


\bigskip
\section{Agenda}\label{sec:Agenda}\index{agenda}
\begin{itemize}
    \item Agenda
	\begin{itemize}
	\myitem Play to find out what happens
	\myitem Make the players'  character's lives not boring
	\myitemend Subvert that which is taken from granted
	\end{itemize}
\end{itemize}
The Agenda is the key thing. If you, as GM, are in doubt what to do, choose the thing that follows the agenda. It's not a bad guide for players either.

\bigskip
\section{Principles}\label{sec:Principles}\index{principles}
\begin{itemize}
	\item Principles
	\begin{itemize}
	\myitem don't waste your players' time
	\myitem the dungeon pressures to conform
	\myitem take tropes to their logical, nonsensical extreme
	\myitem sprinkle details of everyday fantasy everywhere 
	\myitem make the dungeon seem fantastically real
	\myitem name everyone, make everyone rational within their role
	\myitem build a bigger dungeon through play, not plot
	\myitem create interesting dilemmas not interesting traps
	\myitem address yourself to the characters not players
	\myitem make your move, but never speak its name
	\myitem ask loaded questions and build on the answers 
	\myitemend sometimes, reflect a question back upon the players 
	\end{itemize}
\end{itemize}
\bigskip
The Principles are what the GM should be doing if they have their mouth open. If she's chewing pizza instead, call her out on it. Exactly how they translate to the fiction will vary on the tone of the game. These rules are supposed to support a game with a strong undercurrent of pathos with the silliness. The situation is ridiculous, because a lot of fantasy is ridiculous, but a lot of the player characters should be sympathetic or relatable as they struggle in this crazy situation.\footnote{A bit like most sitcoms}  Fundamentally, they are outside of the mainstream surface society and trying to retain a sense of identity when it'd be much easier to slide into institutionalisation. The Dungeon only wants to help, to protect them from the weird outside world where there's no roof and much less glowy rocks. The GM represents the dungeon.


\chapter{Player's core}\index{statistics}

You play as an underling, someone or something recently awoken from your role in the broader dungeon. The dungeon ain't to happy about that, by the way. You'll have a playbook with some moves unique to you, some stats to help define how good you are at different things, some \smallcaps{demons} that represent the things you fear and probably some other stuff like a tribe or some mates or a hoard or a soul-sucking artifact of glowy evilness. You also have access to the basic moves. Everyone has them, and they should be pretty useful.

\marginnote{In play, don't make moves. Do stuff, and keep doing stuff until the GM calls for a move. Dice should only hit the table when the stakes are interesting and the outcome uncertain. You don't need a \smallcaps{roll+thief} to use a doorhandle. You don't even need it to pick a lock when you've got all afternoon and someone making you cups of tea. When you're trapped in a corridor with fire elementals drifting towards you from both sides and your buddy is bleeding out all over your feet, yeah, then you need to roll.} 

There are six main stats in this game.
\begin{itemize}
\item \smallcaps{FIGHTER} - how good you are with violence or raw strength
\item \smallcaps{THIEF} - how good you are with cunning and delicate or precise skill
\item \smallcaps{WIZARD} - how good you are brainpower, thinking or arcane magic stuff
\item \smallcaps{BARD} - how good you are with charm and social connections
\item \smallcaps{Identity} - how good you are at remembering who you actually are beneath the stereotypes.
\item \smallcaps{Health} - how much more damage you can take, for now
\end{itemize}

They'll get set in your playbook. Different underlings have different stats. They may change over play too. Identity certainly will. Unless the move specifically says so, none of those five stats can go above +3 (or below -2, if you are a masochist). A move that uses a stat will state something like \smallcaps{roll + stat name}. That means roll two  normal six-sided dice and add the total to your stat. If you have \smallcaps{wizard} of -1, and you roll a 2 and a 5 the total is 6. Most rules use the format of get 6 or below total and you've messed up, get 7-9 and you succeed, but at a cost. Rolling 10+ is a success, sometimes with a bonus. Try to do that.
Sometime you'll see things like a +1 forward. That means you get to add +1 to your next roll. In the case above, that'd be enough to turn the 6 into 7 and the dangerous failure into a dangerous success.



\smallcaps{health} is the sixth stat and typically starts at nine and fluctuates wildly if you are playing hard enough. You heal up 3 health in a \smallcaps{long rest} - which means a night\footnote{how do underlings in your dungeon keep track of time?} in the fiction, or probably the gap between sessions in real life. Don't hoard your health, we'll cover running out under basic moves. \index{health}

During play, or possibly starting out, you'll pick up items of various use. The 'item' is the basic unit of currency in the dungeon, those pesky heroes keep dropping them when they die and they make handy bartering freebies down the goblin market \index{items}.  The playbook is also a good place to record favours, debts and stuff you want to go back to later. It's also the place to record hobbies and Xp. 

\smallcaps{Hobbies}, in this game, are important. They keep you grounded and bolster your sense of identity. Every time you take the time to act out your hobby, you mark a little tick next to it. Every three ticks means +1 to your identity stat.  \index{hobbies}

\smallcaps{Xp} stands for Experience and represents  how far down the path to the extreme eXpression of your stereotype you've walked, lurched or slithered. Collecting \smallcaps{Xp} will unlock more moves, stat bonuses, moves from other playbooks or more major character development. These advances will be shown in the playboook.

\smallcaps{objectives vs damage} \index{objectives vs damage}. In the basic moves below, commonly you will be told you've succeeded at your objective OR you deal damage. There are two intentions here. The first is to allow a move to be used for different problems where damage is not appropriate: "I want to climb out of the pit", "I want to pick the lock", "I want to study the runes, see if I can find a clue to the ritual." The second intention is to clarify how much damage you do if your objective is something like "I move to stab her in the back," "I exhale and shoot an arrow at that gap in his armour.". Basically, you can't state your objective is to 'behead the world-turtle' and succeed when Great A'Tuin still has a few thousand hitpoints left.\footnote{And the corollary is that the GM should note that relying on hitpoints is bad challenge design. You should be building a jungle-gym not a treadmill. There's an excellent essay for Dungeon World called "The 16Hp Dragon" that provides some good rules of thumb.} As a rule, the \smallcaps{fighter} should be able to deal the most damage to a single target, the \smallcaps{thief} can spread it around but might go through a lot of vials of poison\index[stuff]{poison in vials} or throwing knives\index[stuff]{throwing knives} and the \smallcaps{Wizard} is an unstable powderkeg of potential.
\bigskip

\section{Help or Interfere}\index{help}\index{interfere}

If you want to help or interfere with another player's move, say what you are doing and take that as the objective for a basic move. If you succeed, you can gift them a +1 to their roll, or 'gift' them a -2, if you are hindering them.This stacks with the +1 option in the \smallcaps{Wizard} move. If it is player character vs player character, you don't both \smallcaps{Roll+Fighter} to see who wins. The first person within the fiction to do something that triggers the move does the move, the other person hinders if they want, resolves their move, then the first player finishes the move and then the fiction continues.
Why would you do this instead of doing something for yourself? Sometimes failure is not an option (They're coming, Urg, Steve the  Tortoise, break the door down NOW!), or perhaps you are helping them pull off a unique playbook move. An example might be "I throw sand\index[stuff]{sand} in hero's eyes to allow the Brute to get his "Couple of Mates" into the action. Rather them then me!"


\section{FIGHTER: basic move}\index{Fighter: basic move}
When you try to solve a problem with violence, strength or sheer athleticism \smallcaps{Roll + Fighter}:
\footnote{This is the stat to roll for breaking down doors, taking blows on a shield\index[stuff]{shield} or arm-wrestling in the ork canteen. There is some blurriness with \smallcaps{thief}. Stabbing someone in a fight is a fighter move, but stabbing them in the back is a thief. Shooting a charging hero is direct violence and typically therefore \smallcaps{fighter}, but trying to pull off a trick shot that pins their dagger\index[stuff]{dagger} hand to the wall might be \smallcaps{Thief}. The GM is encouraged to be relaxed about this, but encourage thought about the consequence of a miss. If players are pushing \smallcaps{thief} to try and avoid blowback damage, the GM should hit them with some very hard direct moves when a golden opportunity arrives. Players. I hope you read this.}
\begin{itemize}
\item On a 7+ you succeed at you objective OR deal two damage.
\item On a 10+: choose three
	\begin{itemize}
	\myitem You don't take damage
	\myitem An ally dosen't take damage
	\myitem Something nearby dosen't hear you
	\myitem Your success dosen't make things worse
	\myitemend You don't trigger or prime a trap
	\end{itemize}
\item On a 7-9 choose one and the the GM moves on one
\item On a 6- the GM moves on two.
\end{itemize}

\bigskip




\section{THIEF: basic move}\index{Thief: basic move}
When you try to solve a problem with cunning, treachery or precise and delicate application of skill \smallcaps{roll+ thief}
\footnote{This is the stat to roll for stealth, or dodging arrows\index[stuff]{arrows} too. When it comes to lies and blather, there is some overlap with Bard moves. The GM is encouraged to be relaxed about this, but keep an eye on the consequences of a miss. }
\begin{itemize}
\item on a 7+ you succeed at your objective OR deal one damage\footnote{if your objective was to deal damage, then deal damage.}
\item on a 10+ choose three from below
\begin{itemize}
	\myitem Deal one damage to some other target
	\myitem You do it quickly
	\myitem You can get away cleanly
	\myitem It can't be traced to you
	\myitemend You don't use up an item
	\end{itemize}
\item on a 7-9 choose one, then the GM chooses a remaining option to move against.
\item On a 6- the GM may choose two options to move against.
\end{itemize}
\bigskip

\section{WIZARD: basic move}\index{Wizard: basic move}
When you try to solve a problem with raw brainpower, knowledge or magic \smallcaps{roll+wizard}.
\begin{itemize}
\item On a 7+ you succeed at your objective OR you cast a damaging spell of raw magic\index[stuff]{magic, raw}.\footnote{This represents basic, almost instinctual attacks. Detailed wizard spells are in the playbooks. The difference is like hitting someone with a ripped open electricity wire and constructing a freezer, or a drill or a light bulb that can take the electricity and make something new with it. The Objective might be to expose a weakness, flaw or demon, learn three words about the target or boost an ally}
\item On a 10+ you succeeded at your objective and give \smallcaps{+1 forward to an ally} OR deal one damage to a target. Now choose three:
	\begin{itemize}
	\myitem Leaking magic does not change the environment
	\myitem Something does not accidentally becomes alive
	\myitem Damage doesn't splash to everyone in the room
	\myitem The effect isn't helpfully ironic
	\myitemend The effect stops when you want it to. 
	\end{itemize}
\item On a 7-9 as a 10+ choose one and the GM moves on a remaining option.
\item On a 6- the GM may choose two options to move against.
\end{itemize}
\bigskip



\section{BARD: basic move}\index{Bard: basic move}
When you try to solve a problem with charm, social connections or distraction, \smallcaps{roll+bard}:
\begin{itemize}
\item On a 7+ you succeed at your objective
\item On a 10+: that's it, you succeeded. Deal with the consequences.
\item On a 7-9 choose two:
	\begin{itemize}
	\myitem You distracted an ally too\footnote{be careful with player agency here. It can be be a fun running gag, but make sure the other player is on board. }
	\myitem The target becomes obsessed with you
	\myitem You owe someone. 
	\myitem Someone will come after you, later
	\myitemend They offer you a further opportunity, with a catch
	\end{itemize}
\item on a 6- the GM chooses another player who chooses one for you.
\end{itemize}

\bigskip

\section{IDENTITY: basic move}\index{Identity: basic move}
When you draw on knowledge of the dungeon, or wake after a \smallcaps{long rest}, the dungeon presses in. You, as player, can say the thing you, as character,  know because of your role or job in the dungeon. You get that statement free. You can ask the GM to answer it, or just say something that fits the story. The GM needs to do their best to make it true. Now \smallcaps{roll + identity}
\begin{itemize}
	\item On a 10+ Describe a hobby\footnote{Ideally this should be something outside of your stereotypical job. It's something you value and think about in your spare time and makes plans for. 'Three different times' doesn't mean three cups of tea\index[stuff]{teacups} satisfies a 'tea party' hobby for a Brute. It has to cost something to mean something}. Add it to your playbook. Gain +1 Identity when you have acted on that hobby three different times.
	\item On a 7-9you keep a sense of identity but the dungeon presses on you. Choose one:
	\begin{itemize}
	\myitem I loved that part of my job
	\myitem I hate that part of my job
	\myitemend I had a fierce rival in my job 
	\end{itemize}
	\item On a 6- you slip back into your role. Mindlessly do what your job demands until obstructed or you fail a roll.
\end{itemize}
\bigskip

\section{DEMONS: basic move}\index{demons}
When you face one of your three demons, \smallcaps{Roll +identity}
\begin{itemize}
\item On a 10+: you see how the dungeon is trying to get at you. \smallcaps{mark xp}
\item On a 7-9: if you survive, take \smallcaps{+1 Identity}
\item On a 6-: you panic. Choose flight, fight and make an appropriate basic move.
\end{itemize}

\bigskip


\section{ONE FOOT IN THE GRAVE: Basic move}\index{One Foot in the grave: basic move}\index{dying: basic move}
When your health hits zero choose one:
\begin{itemize}
\item Die free.
\item Come Back: The dungeon isn't done with you. Take -1 to your two highest stats, and wake up later with 3Hp and holes in your memory.
\item Undergo a sea change: The dungeon has a new role for you. Wake up later with a new playbook as the story demands.\footnote{\smallcaps{undead} or \smallcaps{creep} are obvious, but hideous experiments\index[stuff]{hideous experiments} could bring you back as a \smallcaps{Hybrid, Brute, Beast} ect.} Unless the GM feels merciful assume all your items are stolen while you are dead to the world. You keep one hobby and one move but switch stats, including identity. 
\item 'He who kills monsters': where it makes sense, you may come back as the NPC who killed you. Choose an appropriate playbook that's not in use and get with the GM.
\end{itemize}





\chapter{Playbook Summary}

What sort of game do you all want? Talk amongst yourselves.
\bigskip



Want to smash stuff? try the Brute\\
Want to hold a lot of keys and blackmail other players? try the Creep\\
Want to play as a formidable threat? try the Undead\\
Want to play a bunch of characters with sadistic glee? Try the Horde\\
Want to explore separation and back story? Try the Beast\\
Want to play a quieter support role? Try the Construct\\
Want screw with the physics of the world? Try the Magus
Want to include the surface town and be a fun villian? Try the Dregs
Want to explore issues of identity: Try the Hybrid
Want to play quietly but steer the story direction? Try the Plant
Want to play a loud and funny support role? Try the Fey
Want to bring 'other' powers into the story? Try the  Host, 


Still to come:
Elemental Ooze 

%% --------------------------------------------------------

\chapter{Brute}\index{Brute playbook}

Them? They were dead when I got 'ere. Natural shape for dwarves is that, two little shoes poking out of a helmet. Sniff.

\marginnote{Name: Thoth, Ug, Charlie, Caliban, Schtump, Stoor, Begbie\bigskip}
\marginnote{Appearance (Pick 1-3): Hunchback; horns, no neck; huge shoulders; knuckle dragger; (broken)manacles; kilt; tattoos\bigskip}  
\marginnote{Statistics: Start with \smallcaps{Fighter +3}. +1 and -1 to any of the others.\bigskip}
\marginnote{Job (Pick 1): Guard; Gladiator; Miner; Chef\bigskip}
\marginnote{Demon (Pick 3): Weakness; Stupidity; Chains\index[stuff]{chains}; Accidental Harm; Electricity\index[stuff]{electricity}\bigskip}
\marginnote{Hobbies (fill in during play): 
\begin{itemize}
\item [][][][] 
\item [][][][] 
\item [][][][] 
\end{itemize}
\bigskip}

\marginnote{Experience
\begin{itemize}
\item [][][][] add a Brute move
\item [][][][]+1 to any stat
\item [][][][] add a Brute move
\item [][][][]+1 to any stat
\item [][][][] add a move from another playbook
\item [][][][] add a Brute move
\end{itemize}}


Start with two moves.

\section{Gut Feeling} \index{Gut Feeling: move}
When you witness a successful \smallcaps{fighter} move then you can \smallcaps{roll + wizard}:
\begin{itemize}
\item On a 10+ you may ask two of the GM or the player.
\item On a 7-9 you may ask one:
	\begin{itemize}
	\myitem What motivates this fight?
	\myitem What would they kill for?
	\myitem What demon hunts them?
	\myitem What must I beware of?
	\myitemend Who trained them?
	\end{itemize}
\item On a 6- you join the fight instead.
\end{itemize}

\section{Roaring Charge} \index{Roaring Charge: move}
When you charge into another player's fight you may destroy the weapon you are holding to get +1 Fight (to a max of +4) for the scene.

\section{Couple of mates}\index{Couple of Mates: move}
You have a couple of stupid and unruly mates. Name them. Name the body part that drives them.
When you need them to do anything other than join a fight\footnote{sometimes including stopping again} then \smallcaps{roll + bard}
\begin{itemize}
\item On a 10+ they do it just like you asked.
\item On a 7-9 choose one:
	\begin{itemize}
	\myitem They do it, but really badly
	\myitem They do it, but need bribing
	\myitem They do it, but will take it out on an ally later
	\myitemend They obey like the dungeon demands. Take -1 Identity.
	\end{itemize}
\item On a 6- they do their own thing.
\end{itemize}

\section{It's a Weapon!}\index{It's a weapon: move}
When you have taken damage and are weaponless, \smallcaps{Roll + fighter}

\begin{itemize}
\item On a 7+ you find something to use as a weapon
\item On a 10+ choose one:
	\begin{itemize}
	\myitem Improvised shield - you or an ally may ignore the next damage
	\myitem Improvised missile - throw it to deal one damage and take the target out of the fight until you move again.
	\myitemend Puny elf - you overpower one enemy and use them as a weapon to hit others.
	\end{itemize}
\item On a 6- you find something, but can't get to it quickly enough
\end{itemize}


\section{Scarred}\index{Scarred: move}
When you take damage and reach your last health (1Hp), \smallcaps{mark Xp} and resolve any new advances immediately. 


%% ---   ---   ---   ---   ---   

\chapter{Creep}\index{Creep: playbook} % formerly Shadowkin

Sure baby, I can make you feel better. Just let me hold you and pull up a warm and fuzzy memory. Ain't that better?

\marginnote{Name: Legionella; Peez; Latocula; Dalv; Fork-tounge; Shade-of-Grey \bigskip}
\marginnote{Appearance (choose 1-3): Rippling cloak; glowing eyes; bony fingers; scissors; tendrils; smokey hair; blank face\bigskip}
\marginnote{Statistics: \smallcaps{Wizard +2 Thief +1}. +1 and -1 to any of the others.\bigskip}
\marginnote{Job(Pick 1): Jester; Informant; Messenger; Power Source\bigskip}
\marginnote{Demon(Pick 3): Forgetting; Being Forgotten; Guilt; Hunger; Light\index[stuff]{light}\bigskip}
\marginnote{Hobbies (fill in during play): 
\begin{itemize}
\item [][][][] 
\item [][][][] 
\item [][][][] 
\end{itemize}
\bigskip}
\marginnote{Experience
\begin{itemize}
\item [][][][] add a Creep move
\item [][][][] +1 to any stat
\item [][][][] add a Creep move
\item [][][][] +1 to any stat
\item [][][][] add a move from another playbook
\item [][][][] add a Creep move
\end{itemize}}

Start with two moves and a chorus

\section{Chorus}\index{Chorus}
A creep is always accompanied by a \smallcaps{chorus} -  a collection of soul fragments\index[stuff]{soul fragments} sustained by energies leaking through the hole in the world that the creep represents. The chorus members are fragments patched together almost randomly - faces squished together, rats and bat spirits filling holes in skin or missing body parts. The chorus are simple minded, fixated on pet topics and without agency.

\section{Commune}\index{Commune: move}
When you ask a question of the chorus \smallcaps{roll +wizard}
\begin{itemize}
\item On a 10+ ask two questions from the list below
\item On a 7-9 ask one question.
	\begin{itemize}
	\myitem Who died here?
	\myitem What was the last thing they saw?
	\myitem Where is a dangeours item?
	\myitem Who is lying about their memories?
	\myitemend Where is the person this body part belongs to?
	\end{itemize}
\item On a 6- you may still ask a question, and the chorus will reply truthfully, but the answer comes in the form of a riddle, eerie song or a crude cartoon in blood on the wall.
\end{itemize}

\section{Brain slither}\index{Brain slither: move}
When you break somebody's skin, \smallcaps{roll + Wizard}:
\begin{itemize}
\item On a 10+ you steal their shadow\index[stuff]{shadow} and hold it indefinitely. You can release that shadow at any time to summon one of their demons. You decide which demon.
\item On a 7-9 as for the 10+, but the GM will decide which of their demons is released.
\item On a 6- their demon comes looking for you both
\end{itemize}

\section{Black Leech}\index{Black leech: move}
When a memory is killed or lost to the world and you add them to the chorus \smallcaps{roll+wizard}.
\begin{itemize}
\item On a 10+ choose three from the list below
\item On a 7-9 choose two:
	\begin{itemize}
	\myitem You can teleport between shadow\index[stuff]{shadow}
	\myitem You can create a pool of darkness\index[stuff]{darkness}
	\myitem You can fill a room with silence
	\myitemend You don't loose an identity point (-1 to Identity otherwise)
	\end{itemize}
\item these powers last until you next loose health
\item On a 6- the fragment tears free and flies off into the dungeon as a seedling rival creep. Take one damage
\end{itemize}

\section{Grey Matter}\index{Grey Matter: move}
When you wrap around someone \smallcaps{Roll + Thief}:
\begin{itemize}
\item On a 10+ choose three from the list below
\item On a 7-9 choose two:
	\begin{itemize}
	\myitem Temporarily increase their \smallcaps{Wizard} by +1
	\myitem You can enter their dreams and memories\index[stuff]{dreams and memories}
	\myitem They can use your powers from \smallcaps{Black Leech}
	\myitemend You don't heal them one health at the cost of two damage to yourself.
	\end{itemize}
\end{itemize}

\section{Greek Chorus}\index{Greek chorus: move}
At the end of a session where you entertained the table by playing the chorus voices too \smallcaps{mark Xp}


%% ---   ---   ---   ---   ---   
\chapter{Undead}\index{Undead: playbook}

 I've waited a long time. The only way to power in this place is a dead man's shoes\index[stuff]{shoes}.

\marginnote{Name:Azalgot; Sssinter; Kn'kle; Leech; Vertebran \bigskip}
\marginnote{Appearance (choose 1-3): Skeletal; rotting corpse; wrappings; glowing runes; armoured; pierced; crawling with maggots\bigskip}
\marginnote{Statistics: \smallcaps{Wizard +2 Fighter +1}. \bigskip}
\marginnote{Job(Pick 1): Honour Guard; Sentinel; Smith; Channeler \bigskip}
\marginnote{Demon(Pick 3): Chaos; Being Human; Logical paradoxes; Growth\index[stuff]{growth, verdancy}; Theft\bigskip}
\marginnote{Hobbies (fill in during play): 
\begin{itemize}
\item [][][][] 
\item [][][][] 
\item [][][][] 
\end{itemize}
\bigskip}
\marginnote{Experience
\begin{itemize}
\item [][][][] add an Undead move
\item [][][][] +1 to any stat
\item [][][][] add an Undead move
\item [][][][] +1 to any stat
\item [][][][] add a move from another playbook
\item [][][][] add an Undead move
\end{itemize}}

Start with two moves and two artifacts

\section{Artifacts}\index{Artifacts}
At the start of the game choose two different artifacts. Add a descriptive tag (or get with the GM for to customise further) As long as the artifact is in your possession you gain that statistic bonus. If the artifact is out of your possession you have a sense of roughly where it is.:
\begin{itemize}
\item Choose two different rows:
	\begin{itemize}
	\myitem Shield \smallcaps(+1 Fighter) \index[stuff]{shield}
	\myitem Spear \smallcaps(+1 Fighter) \index[stuff]{spear}
	\myitem Staff/Crown/Necklace \smallcaps(+1 Wizard) \index[stuff]{staff} \index[stuff]{crown} \index[stuff]{necklace}
	\myitem Instrument \smallcaps(+1 Bard) \index[stuff]{intrument}
	\myitemend Undead Pet \smallcaps(+1 Thief)
	\end{itemize}
\item Choose a tag from each row 
	\begin{itemize}
	\myitem Religious; Bone; Compact; Fine; Bronze
	\myitemend Cursed; Golden; Large; Ancient; Delicate
	\end{itemize}
\end{itemize}

\section{Reservoir of Power}\index{Reservoir of Power: move}
In a high stakes situation where the artifact's tag may help, you may double that artifact's stat bonus and choose one:
\begin{itemize}
\item Someone recognises the artifact, and wants it
\item The reservoir is exhausted. You can't do this again until the next long rest 
\item The dungeon presses in
\end{itemize}

\section{Necromancy}\index{Necromancy: move}
When you utilise necromancy \smallcaps{ Roll + Bodies ritually prepared}: 
\begin{itemize}
\item On a 10+ Get three hold
\item On a 7-9 Get two hold. Spend hold one for one to:
	\begin{itemize}
	\myitem heal 1 damage to yourself
	\myitem Raise a short lived minion (lasts until next long rest)
	\myitem Blind someone with terrifying visions
	\myitemend Hold a room fast with rotting limbs for a scene
	\end{itemize}
\item On a 6 one of the bodies rises again as an enemy
\end{itemize}

\section{Rise Again}\index{Rise Again: move}
When you draw on dark magic to ignore damage \smallcaps{roll+wizard}
\begin{itemize}
\item On a 10+ Get three hold
\item On a 7-9 Get two hold. Spend hold one for one to:
	\begin{itemize}
	\myitem reduce damage received by one
	\myitem take the damage now, but you can piece yourself back together afterwards
	\myitem they'll mistake you for dead if you lie still
	\myitemend you don't desperately crave blood, brains or flesh. Otherwise you must satisfy this hunger before you can make this move again.
	\end{itemize}
\item On a 6 take the damage and the hunger
\end{itemize}

\section{Armoury}\index{Armoury: Move}
When you lend another player an artifact they don't get the same bonus but may reroll a failed roll that uses the appropriate stat. You can't reroll a reroll.

\section{All is Dust}\index{All is Dust}
When you use a player character or a friendly NPC's dead body in a necromantic ritual mark Xp.



%% ---   ---   ---   ---   ---   
\chapter{Horde}\index{Horde}

Hey Grog! You owe me a favour. I bet Skibl three squid he couldn't run across the backs of those clockwork alligators faster then Mibl, I bet Mibl the same and they both got eaten. I told Gnolly you were the best clockwork alligator repair gremlin in the buisness, and I reckon that now you've got that contract, you owes me a favour. Now I needs to get into Gnolly's safe....

\marginnote{Name: Scumpit; Noerag; Hurly; Dai Drawpaw; Gnolly \bigskip}
\marginnote{Appearance (choose 1-3): Scaly; slimy; hairy; well-dressed; hat-wearing; long fingered\bigskip}
\marginnote{Statistics: \smallcaps{Thief +3}.  +1 and -1 to any of the others \bigskip}
\marginnote{Job(Pick 1): Tinker; Herder; Hawker; Dung-farmer \bigskip} 
\marginnote{Demon(Pick 3): Loneliness; Cannibalism; Mother-in-Law; Predators; Disease\index[stuff]{disease}\bigskip}
\marginnote{Hobbies (fill in during play): 
\begin{itemize}
\item [][][][] 
\item [][][][] 
\item [][][][] 
\end{itemize}
\bigskip}
\marginnote{Experience
\begin{itemize}
\item [][][][] add a Horde move
\item [][][][] +1 to any stat
\item [][][][] add a Horde move
\item [][][][] +1 to any stat
\item [][][][] add a move from another playbook
\item [][][][] add a Horde move
\end{itemize}}

Start with two moves and the Bickering Swarm

\section{The Bickering Swarm}\index{The Bickering swarm}
Members of the Horde are rarely alone. At anyone one time there's a member of the horde present for every two health points you have (rounding up). When you take two damage, someone just died, hopefully both gruesomely and entertainingly. When you heal two health points, make a note and have a cousin or niece of the current Face of the Horde turn up out of a chimney, toilet, locked box or similar when convenient. You aren't the leader of the entire horde, but whichever member you are currently focusing on as the the face for your character is probably quite well respected. By Horde standards anyway.

\section{Maslov's Challenge}\index{Maslov's Challenge: move}
At the start of the session \smallcaps{roll+thief} on behalf of the wider horde to see what they are dealing with. Whatever the challenge, if you help \smallcaps{mark Xp}:
\begin{itemize}
\item On a 10+,  gain a reroll in your pocket for this session and choose one challenge:
	\begin{itemize}
	\myitem a wedding
	\myitem a trade war
	\myitemend a lost child
	\end{itemize}
\item On a 7-9  gain a reroll in your pocket for this session and choose one problem:
	\begin{itemize}
	\myitem a predator
	\myitem bandits
	\myitemend a flood
	\end{itemize}
\item On a 6- choose one desperate need:
	\begin{itemize}
	\myitem food
	\myitem safe rooms
	\myitemend medicine \index[stuff]{medicine}
	\end{itemize}
\end{itemize}

\section{Yes Boss}\index{Yes Boss:move}
When you call on the entire horde to help with something \smallcaps{Roll+Bard}
\begin{itemize}
\item On a 10+ choose two bonuses:
	\begin{itemize}
	\myitem They turn up in big numbers
	\myitem They are well equipped for the task
	\myitemend They'll stick around for a bit, even if there's trouble
	\end{itemize}
\item On a 7-9 choose a bonus from above and complication from below
	\begin{itemize}
	\myitem You owe a massive favour
	\myitem They bring trouble with them
	\myitemend They are undisciplined kleptomaniacs
	\end{itemize}
\item On a 6- choose one bonus but the GM chooses the complication
\end{itemize}

\section{It's a trap!}\index{It's a trap!: move}
When you spend time working to make or understand a trap \smallcaps{Roll+Thief}
\begin{itemize}
\item On a 10+ choose three:
	\begin{itemize}
	\myitem the trigger is reliable and in your control
	\myitem the trap is well hidden
	\myitem the trap resets automatically
	\myitem the pit/ropes stops people escaping \index[stuff]{ropes}
	\myitem the gas/darts has a room wide effect \index[stuff]{gas} \index[stuff]{darts}
	\myitem the trap does not inflict permanent damage
	\myitemend the trap can be dismantled for a useful part
	\end{itemize}
\item On a 7-9 choose two
\item On a 6- it goes hilariously wrong. Take two damage.
\end{itemize}

\section{Gossipmonger}\index{Gossipmonger: move}
When another player character asks you about an NPC \smallcaps{Roll+Thief}
\begin{itemize}
\item On a 10+ ask the GM three:
	\begin{itemize}
	\myitem What was the NPC asking around for at the market?
	\myitem What did they last buy?
	\myitem What do they think is secret?
	\myitem Who are they connected to?
	\myitemend What do they have of value?
	\end{itemize}
\item On a 7-9 ask one
\item On a 6- ask one but be prepared for the worst
\end{itemize}

\section{Filofax}\index{Filofax: move}
You know the Horde, and the Horde knows everyone. When you want to hire someone, name them, describe them and \smallcaps{roll+items in payment}
\begin{itemize}
\item On a 10+ choose three:
	\begin{itemize}
	\myitem They are scary powerful
	\myitem They won't ask for extra payment
	\myitem They are available immediately
	\myitemend They don't have a grudge against a player character
	\end{itemize}
\item On a 7-9 choose two
\item On a 6- the message gets confused. The GM will say who you get instead.
\end{itemize}



%% ---   ---   ---   ---   ---   
\chapter{Beast}\index{Beast: playbook}

Big and old and alone. For ages past. For now.

\marginnote{Name: Aaargh; the great clawed beast; smell-in-the-dark; white-wyrm-o-the-well; Gulp; Bessie; Danke \bigskip}
\marginnote{Appearance (choose 1-3): extra legs, albino, bedraggled, too many eyes; prehistoric; warty; sleek; corpulent\bigskip}
\marginnote{Statistics: \smallcaps{Fighter +2, Thief +1, Health+3}. Your max health is 12 \bigskip}
\marginnote{Job(Pick 1): Trophy pet; Native; Fresh Meat; Haulage \bigskip}
\marginnote{Demon(Pick 3): Fire; suffocation; open spaces; master's displeasure; bells\bigskip} \index[stuff]{bells}
\marginnote{Hobbies (fill in during play): 
\begin{itemize}
\item [][][][] 
\item [][][][] 
\item [][][][] 
\end{itemize}
\bigskip}
\marginnote{Experience
\begin{itemize}
\item [][][][] add a Beast move
\item [][][][] +1 to any stat
\item [][][][] add a Beast move
\item [][][][] +1 to any stat
\item [][][][] add a move from another playbook
\item [][][][] add a beast move
\end{itemize}}


Start with two moves:

\section{Lifecycle}\index{Lifecycle: move}
You have a countdown stat. It starts at 35. You can reroll a failed roll but the total of the new roll is subtracted from the countdown. When the countdown hits zero you will (choose two):
\begin{itemize}
\item die of old age
\item give birth
 \item need to return to the spawning pool \index[stuff]{spawning pool}
\item crave a mate
\item enter a mindless hunger frenzy
\item change into a radically different form \marginnote{think tadpole to frog, caterpillar to butterfly or the way a flatfish turns sideways and their eyes migrate to the same side of their head. It's natural and important and maybe scary for you, but should really change how your companions interact with you}
\item change sex
\end{itemize}

\section{Unleash instincts}\index{Unleash instincts: move}
When you are in fear of your life \smallcaps{Wizard +1}



\section{Arcane Bridle}\index{Arcane Bridle: move}
\marginnote{optional: How do you communicate? Pick two
\begin{itemize}
\item with noises
\item with body language
\item with limited speech
\item with telepathy
\item with magic auras
\item with a tap code
\end{itemize}}

You have a movement skill suited to your environment (choose one when you create your character)
\begin{itemize}
\myitem flyer
\myitem climber
\myitem swimmer and diver
\myitem burrower
\myitemend leaper
\end{itemize}
Someone trained you as a mount (who?). If a player character orders you to carry them, treat as an automatic 7-9 result. Otherwise \smallcaps{roll+Identity}:
\begin{itemize}
\item On a 10+ you move well together. The rider can carry out actions while riding you and can only be hit by ranged attacks
\item On a 7-9 they are safe while on you. They can only be hit by ranged attacks
\item On a 6- you remember the beatings. Your rider will fall off at the worst possible time.
\end{itemize}

\section{Cat and Mouse}\index{Cat and Mouse: move}
\marginnote{optional: How do you feed? Pick one
\begin{itemize}
\item stalking prey
\item lure and trapping
\item chase and kill
\item carrion eater
\item scavenging bully
\item rubbish picker
\item wandering grazer
\item gather and store
\item ambient flow filter
\end{itemize}}

You remember being hunted before (by who?). Whether you are the cat or the mouse now, \smallcaps{roll + thief}:
\begin{itemize}
\item On a 10+ get all three
	\begin{itemize}
	\myitem You can choose the moment of ambush
	\myitem Another player character won't get hurt
	\myitemend You will be the cat this time
	\end{itemize}
\item On a 7-9 choose one
\item On a 6- choose one, but you are left very vulnerable to a new threat
\end{itemize}

\section{Around and About}\index{Around and about: move}
At the beginning of the session \smallcaps{roll+thief}
\begin{itemize}
\item On a 10+ hold one+1
\item On a 7-9 hold one
\item At any time you or the GM can spend your hold to have you already be there, with or without clear explanation. If your hold was one+1, take +1 to your next move now. Mark Xp when the hold is used.
\item On a 6- the GM holds one and can spend it to have you there already but somehow pinned, caught or trapped.
\end{itemize}



%% ---   ---   ---   ---   ---   
\chapter{Construct}\index{Construct: playbook}

Dolem is. Dolem does. Dolem thinks. Dolem isn't. 

\marginnote{Name: Stone; Furnit; Dresser; Crank; Mallard \bigskip}
\marginnote{Appearance (choose 1-3): polished; china; clockwork; runic panels; stonework; rubble; \bigskip}
\marginnote{Statistics: \smallcaps{Thief +2, Fighter +1}.  +1 and -1 to any of the others \bigskip}
\marginnote{Job(Pick 1): Savant; Servant; Hunter; Quartermaster \bigskip}
\marginnote{Demon(Pick 3): Replacement; Corrosion; Lies; Mirrors; Innocence destroyed\bigskip} \index[stuff]{mirror} \index[stuff]{corrosion}
\marginnote{Hobbies (fill in during play): 
\begin{itemize}
\item [][][][] 
\item [][][][] 
\item [][][][] 
\end{itemize}
\bigskip}
\marginnote{Experience
\begin{itemize}
\item [][][][] add a Construct move
\item [][][][] +1 to any stat
\item [][][][] add a Construct move
\item [][][][] +1 to any stat
\item [][][][] add a move from another playbook
\item [][][][] add a Construct move
\end{itemize}}

\section{Astral Projection}\index{Astral Projection: move}
When you deal with an exciting unknown \smallcaps{roll+wisdom}:
\begin{itemize}
\item On a 10+ you may state three things about the unkown's future
\item On a 7-9 you can state a single thing.
\item The GM must make a good faith effort to have those things come true
\item On a 6- you may state one thing. The GM will later reveal why it is true and bad for you.
\end{itemize}

\section{Transformation}\index{Transformation: move}
When you want to mimic some everyday fantasy dungeon object \smallcaps{roll + Thief}:
\begin{itemize}
\item On a 10+ choose two:
	\begin{itemize}
	\myitem You cannot be spotted (otherwise you leave a telltale)
	\myitem You can transform in the blink of an eye (otherwise it takes some time)
	\myitemend You can go up or down a size when you transform (otherwise you remain the same size and weight)
	\end{itemize}
\item On a 7-9 choose one.
\item On a 6- your disguise is poor, but you can do something before you are spotted if you move quickly.
\end{itemize}

\section{Elemental Engine}\index{Elemental Engine: move}
When you consume a large quantity of an appropriate fuel, \smallcaps{roll + thief} \index[stuff]{fuel}
\begin{itemize}
\item On a 10+ hold 3
\item On a 7-9+ hold 1
\item On a 6- hold 1 but suffer appropriate symptoms of indigestion at the worst possible time.
\item spend hold 1 for 1 as required for a scene
	\begin{itemize}
	\myitem \smallcaps{+1 Fighter} Consume flammable fuel or metal items \index[stuff]{metal}
	\myitem \smallcaps{+1 thief} Consume fabrics, inks or poisons \index[stuff]{fabric} \index[stuff]{ink} \index[stuff]{poisons}
	\myitem \smallcaps{+1 Wizard} Consume unstable magical or radioactive items \index[stuff]{radioactive items} 
	\end{itemize}
\end{itemize}

\section{Simulacrum}\index{Simulacrum: move}
When you pretend to be an NPC you know well enough to simulate \smallcaps{roll+thief}
\begin{itemize}
\item On a 10+ you get all three:
	\begin{itemize}
	\myitem Another player character can ask what the the NPC wants and you can give an detailed answer \footnote{Collaborate with the GM, but the answer should be accurate}
	\myitem You can stop the simulation at will (otherwise it takes a bit of time or might come back)
	\myitemend The simulation doesn't act on it's own desires 
	\end{itemize}
\item On a 7-9 choose 1
\item On a 6- You get the simulation up and running, but something else happens before you can use it, or turn it off.
\end{itemize}

\section{A million monkeys}\index{A million monkeys: move} \index[stuff]{paper} \index[stuff]{ink}
When you have time, safety and your weight in paper and ink, you may write continuously for a day and night. For this feat of endurance, \smallcaps{roll+fighter}:
\begin{itemize}
\item On a 10+ you have discovered a new valuable spell
\item On a 7-9 treat the result as a 10+ \smallcaps{Wizard basic move}.
\item On a 6- you discover a new spell, but the results have not become apparent yet
\end{itemize}
l 

%% ---   ---   ---   ---   ---   
\chapter{Magus} % formerly Wizard

"They said I was a madman! A fool! Well, who's the fool now?"

\marginnote{Name: Curly; Ralla; Davos; Thuy; Wavv }
\marginnote{Title: magnificent; wise; summoner; astral; runelord; calligrapher; weft\bigskip}
\marginnote{Appearance (choose 1-3): skinny; bulging eyes; things tied in hair; tattered cloak; pointy shoes; grand hat \bigskip} \index[stuff]{hat}
\marginnote{Statistics: \smallcaps{Wizard +3}.  +1 and -1 to any of the others \bigskip}
\marginnote{Job(Pick 1): Bureaucrat; Hermit; Alchemist; Rival to the old master  \bigskip}
\marginnote{Demon(Pick 3)Old age; ambition; blood; homelessness; betrayal\bigskip} \index[stuff]{blood} 
\marginnote{Hobbies (fill in during play): 
\begin{itemize}
\item [][][][] 
\item [][][][] 

\item [][][][] 
\end{itemize}
\bigskip}
\marginnote{Experience
\begin{itemize}
\item [][][][] add a Magus move
\item [][][][] +1 to any stat
\item [][][][] add a Magus move
\item [][][][] +1 to any stat
\item [][][][] add a move from another playbook
\item [][][][] add a Magus move
\end{itemize}}

start with two moves.

\section{Grand Ritual}\index{Grand Ritual: move}

\marginnote{Magus poke around the basic rules of reality, so a grand ritual might be about reversing gravity, making everyone young again, brain swapping, making something come alive, awakening a dark god, transfusing experience ect ect. The player of the Magus takes on some of the responsibility that is normally the GM's to meet the social contract, to ensure the people around the table are all having a good time, even as their characters are all shrunk to rat size.}

At the start of the session describe your new grand ritual planned effects to the GM. They will say "sure, but first..."
They pick three, stringing them together with an and:
\begin{itemize}
\item It'll take many attempts
\item It needs wild ingredients   \index[stuff]{wild ingredients}
\item It needs travel to .....
\item It needs help from .....
\item It'll be weak or unreliable
\item It'll put you all in serious danger
\item You'll need to make ...... first
\end{itemize}
When you accomplish two thirds of the requirements \smallcaps{mark Xp}. When you hit all three, you can accomplish it as agreed. 

\section{Henchlings}\index{Henchlings: Move}
You have a bunch of creepy and loyal henchlings. When you order them to watch another player character or NPC \smallcaps{roll+Bard}:
\begin{itemize}
\item On a 10+ they report back in time for you to interfere
\item On a 7-9 choose one:
	\begin{itemize}
	\myitem things get messy
	\myitem you  are clearly to blame
	\myitemend it is too late by the time you know
	\end{itemize}
\item On a 6- the target of the watchers chooses one for you
\end{itemize}

\section{Meddler's curse}\index{Meddler's curse: move}
When you curse someone from the depths of your soul \smallcaps{roll + wizard}:
\begin{itemize}
\item On a 10+ the curse sticks, hindering them in an ironic reflection of what they desire most. (GM's or owning player's call)
\item On a 7-9 the curse happens but is short lived
\item On a 6- the curse twists. It attacks what they desire directly.
\end{itemize}

\section{Familiar}\index{Familiar: move}
You have a familiar. Name them, describe them, say what body part drives them. When you let them loose to satisfy their impulse, \smallcaps{mark Xp}. When you send them on a mission, \smallcaps{roll+wizard}
\begin{itemize}
\item On a 10+ hold three
Spend hold one for one for the familiar to use basic moves (their statistics are zero, but they can receive help elsewhere)
\item On a 7-9 hold one
\item On a 6- hold one but take 1 damage as you invest them with more power then intended.
\end{itemize}

\section{Sanctum}\index{Sanctum: move}
You have a sanctum, a workshop, library or pocket universe. Choose four good things and two bad. These are all about fictional positioning, magical equivalents of announcing you climb the wall using crampons and a rope compared to just climbing it. In your sanctum, the things you should try to pull off should be big and dramatic and force the attention of the rest of the dungeon. Otherwise, what's the point?
\begin{itemize}
\item Good things
	\begin{itemize}
	\myitem guardian spirit
	\myitem extensive library \index[stuff]{books}
	\myitem warded prison
	\myitem scrying tools \index[stuff]{tarot cards} \index[stuff]{crystal ball}
	\myitem gold seams \index[stuff]{lumps of gold}
	\myitem herb/fungus garden \index[stuff]{herbs} \index[stuff]{mushrooms}
	\myitem exotic collection \index[stuff]{shrunken heads}
	\myitem comfortable bed \index[stuff]{bed}
	\myitem totems \index[stuff]{totems}
	\myitem small forge \index[stuff]{forge} \index[stuff]{hammer} \index[stuff]{anvil}
	\myitem alchemical glasswork \index[stuff]{glasswork} \index[stuff]{jars} \index[stuff]{chemicals}
	\myitemend deep freeze storage \index[stuff]{ice}
	\end{itemize}
\item Bad things
	\begin{itemize}
	\myitem Cramped for space
	\myitem hard for you to access
	\myitem dangerous neighbours
	\myitem too well known
	\myitem infested
	\myitem coveted
	\myitemend power surges
	\end{itemize}
\end{itemize}

%% ---   ---   ---   ---   ---   
\chapter{Dregs}

"think this place is bad? you should see the town."

\marginnote{Name: Black; Bator; Don; Felix; Esper }
\marginnote{Appearance (choose 1-3): eyepatch; whipmarks; brandings; rich clothes; curled hair; riding cloak \bigskip} \index[stuff]{cloak} \index[stuff]{hair}
\marginnote{Statistics: \smallcaps{Thief+2, Bard+1 }.  +1 and -1 to any of the others \bigskip}
\marginnote{Job(Pick 1): Doorkeeper; Ambassador; Smuggler; Scout  \bigskip}
\marginnote{Demon(Pick 3) Winter; religion; parasites; addiction; the noose \bigskip} \index[stuff]{winter} 
\marginnote{Hobbies (fill in during play): 
\begin{itemize}
\item [][][][] 
\item [][][][] 

\item [][][][] 
\end{itemize}
\bigskip}
\marginnote{Experience
\begin{itemize}
\item [][][][] add a Dregs move
\item [][][][] +1 to any stat
\item [][][][] add a Dregs move
\item [][][][] +1 to any stat
\item [][][][] add a move from another playbook
\item [][][][] add a Dregs move
\end{itemize}}


start with two moves and an unruly gang

\section{Smugglers}\index{Smugglers: move}
When you return from secretly selling items to a contact in town (who?) \smallcaps{roll+items sold}:
\begin{itemize}
\item On a 10+ get three hold. Spend hold one for one to drink your cut of the goods and get \smallcaps{+1 wizard or +1 bard} for the scene.
\item On a 7-9 get one hold
\item On a 6- get one hold but a greedy party of adventurers is tracking you.
\end{itemize}

\section{Bandits} \index{Bandits: move}
When you and your gang launch an ambush on the road \smallcaps{roll+thief} 
\begin{itemize}
\item On a 10+ your preparations are clever. Take \smallcaps{+1 thief} into the scene.
\item On a 7-9 your preparations are crude. Take \smallcaps{+1 fighter} into the scene
\item On a 6- it turns out one of your gang is a traitor
\end{itemize}

\section{Drunks} \index{Drunks: move}
You have a vice, a reason you are not welcome in surface society. Choose one and answer what the GM asks you about it:
\begin{itemize}
\item intoxicating drugs
\item gambling with body parts
\item socially unacceptable food \footnote{this is socially unacceptable to the surface society. It could range from cannibalism to opening your boiled egg at the big end. This is set as much by the tone of the game and the stomach of your table. Be aware of the social contract and don't trample on people's phobias. Refer to \nameref{X-Card} for more guidance.}
\item socially unacceptable relationships
\item a need to speak truth to the powerful
\end{itemize}
When you have indulged in your chosen vice heavily and you fail a roll, accept the effects of the fail, but you can try to succeed on the roll one more time before the story moves on. This means you can have a 10+ result and a 6- result at the same time. \marginnote{needs testing, and possibly switching for a more traditional move. It breaks the conversation pattern (but does mimic a drunks's conversation pattern... Playability is more important than meta-jokes though}

\section{Gambler's tarot} \index{Gambler's tarot: move}
When your gang play dice, cards or bingo and you have them wondering aloud about things, \smallcaps{roll + thief}
\begin{itemize}
\item On a 10+ ask three questions. The GM will answer but ask what it looks like in the cards, dice results ect.: 
	\begin{itemize}
	\myitem How close is my enemy?
	\myitem Where's my path forward?
	\myitem Am I safe right now?
	\myitem What should I be on the lookout for?
	\myitem Where is a tempting target?
	\end{itemize}
\item On a 7-9 ask one
\item On a 6- ask one anyway but you might not like the answer
\end{itemize}

\section{Mischief} \index{Mischief: move}
You have a wild talent, mutation or racial trait. Choose one:
\begin{itemize}
\item People can't break eye contact with you
\item When you are making music, they want to give you gifts
\item When you shake hands, something of theirs appears in your bag
\end{itemize}	


%% ---   ---   ---   ---   ---   
\chapter{Plant}

"You little warm hairy things, always running, always in a hurry. Oh sorry, I was talking to the rat."

\marginnote{Name: Gren; Mush; Sporth; Dryer; Cactus Jack }
\marginnote{Appearance (choose 1-3): tree stump; toadstool; colourful pot; long shoots; lopsided face; mossy hair; bright fruits \bigskip} 
\marginnote{Statistics: \smallcaps{Wizard+2, Bard+1 }.  +1 and -1 to any of the others \bigskip}
\marginnote{Job(Pick 1): Herbalist; Farmer; Librarian; Assassin  \bigskip}
\marginnote{Demon(Pick 3) drought; vegetarians; landslide; butterflies; blades \bigskip} \index[stuff]{drought}  \index[stuff]{butterflies} \index[stuff]{blades} 
\marginnote{Hobbies (fill in during play): 
\begin{itemize}
\item [][][][] 
\item [][][][] 

\item [][][][] 
\end{itemize}
\bigskip}
\marginnote{Experience
\begin{itemize}
\item [][][][] add a Plant move
\item [][][][] +1 to any stat
\item [][][][] add a Pant move
\item [][][][] +1 to any stat
\item [][][][] add a move from another playbook
\item [][][][] add a Plant move
\end{itemize}}

\section{Green Fingers} \index{Green Fingers: move}
You are the guardian of the mystic pool/golden tree/ holy compost heap or similar. When another player character bathes in the pool and you channel the magic \smallcaps{roll + wizard}
\begin{itemize}
\item On a 10+ they regrow a missing limb, gain something for their hobby or achieve total calm
\item On a 7-9 they gain something for their hobby or regrow superfical damage like fingers or hair
\item On a 6- they gain an obsession with plants as a new hobby, or a green mutation.
\end{itemize}

\section{Secret Garden} \index{Secret Garden: move}
When you look for the secrete garden, open a door and \smallcaps{roll+wizard}
\begin{itemize}
\item On a 10+ choose two:
	\begin{itemize}
	\myitem The garden is peaceful and deserted
	\myitem A threat does not follow you in
	\myitemend When you leave the gate opens somewhere new and safe
	\end{itemize}
\item On a 7-9 choose one
\item On a 6- The GM makes a move on any or all of them.
\end{itemize}

\section{Herbaceous} \index{Herbaceous: move}
Your leaves, spores, pods or other minor body parts have an intense experience on others. If they ingest them they'll have a very trippy vision supplied by the GM. Knowledge is knowledge, right?

\section{Dryad Dust} \index{Dryad Dust: move}
When you use your perfume, spores or pheremones to influence a group \smallcaps{roll+bard}.
\begin{itemize}
\item On a 10+ the group is swayed by your words. Player characters get a +1 in their pocket to follow your suggestion
\item On a 7-9 the group is swayed by words but also agitated, passions inflamed or spoiling for a fight (GM to detail)
\item On a 6- the group is agitated, spoiling for a fight. Player characters get +1 to attacks on you.
\end{itemize}

\section{One with the Leaf} \index{One with the Leaf: move}
You have a countdown stat. It starts at 35. You can reroll a failed roll but the total of the new roll is subtracted from the countdown. When the countdown hits zero you will (choose two now):
\begin{itemize}
\item flower, then die
\item implant children in others
\item die back to a tiny version of yourself \footnote{like a plant cutting or a twig. One Hp}
\item find somewhere to hibernate
\item eat, spread and grow voraciously
\item grow extra sexual organs
\item change into a radically different form
\end{itemize}


%% ---   ---   ---   ---   ---   
\chapter{Fey}

"May the road not rise up and hit you in the face: Ancient fey blessing"

\marginnote{Name: Barley; Driski; Grappa; Kalvas; Potchy; Sarky }
\marginnote{Appearance (choose 1-3): slender frame; pot-bellied; extended fingers; huge beard; \bigskip} 
\marginnote{Statistics: \smallcaps{Bard+3 }.  +1 and -1 to any of the others \bigskip}
\marginnote{Job(Pick 1): Judge; Innkeeper; Food-stall; Fisherman  \bigskip} \index[stuff]{fish}
\marginnote{Demon(Pick 3) broken promises; iron; cats; drowning; poverty \bigskip} \index[stuff]{iron}  \index[stuff]{cats} 
\marginnote{Hobbies (fill in during play): 
\begin{itemize}
\item [][][][] 
\item [][][][] 
\item [][][][] 
\end{itemize}
\bigskip}
\marginnote{Experience
\begin{itemize}
\item [][][][] add a Fey move
\item [][][][] +1 to any stat
\item [][][][] add a Fey move
\item [][][][] +1 to any stat
\item [][][][] add a move from another playbook
\item [][][][] add a Fey move
\end{itemize}}

\section{Oracle} \index{Oracle: move}
At the start of the session \smallcaps{roll+number of player characters you are touching}
\begin{itemize}
\item On a 10+ you may ask "What will happen to ..." for each character you are touching
\item On a 7-9+ you can only ask for one character
\item On a 6- The omens are bleak. Everyone involved takes -1 ongoing until a demon is faced.
\end{itemize}

\section{Rumours} \index{Rumours: move}
You have a sideline at a late night tavern. Name three regulars and your best drink.
When other player character buy drinks at your place and ask for rumours regarding someone/something ; choose two and let the GM fill in the details:
\begin{itemize}
\myitem "I know ...  bought it"
\myitem "They've been hanging around with ..."
\myitem "I hear ... is selling, under the counter like."
\myitem "Ach, they did ... a terrible wrong."
\myitemend "They are pulling a stunt at ...'s place"
\end{itemize}

\section{Bindings} \index{Bindings: move}
When someone breaks a promise to you, name them and two other NPCs and \smallcaps{roll+bard}
\begin{itemize}
\item On a 10+ choose two:
	\begin{itemize}
	\myitem They'll get involved in a fight
	\myitem The target will be stolen from
	\myitemend Magic will keep pushing them together
	\end{itemize}
\item On a 7-9 Choose one:
\item On a 6- Choose one, but the bindings go both ways. The Target may choose 1 with you as the target.
\end{itemize}

\section{Epic Tales} \index{Epic Tales: move}
When you narrate your thoughts and actions as though oyu were the star of an epic tale, you can distract reality by your words. Any roll totalling less then 5 may be rerolled., although reality sticks the second time.

\section{Special Reserve} \index{Special Reserve: move}
When you serve something special from your supplies \smallcaps{roll+wizard}:
\begin{itemize}
\item On a 10+ they are immune to one thing for the next scene. Specify it.
\item On a 7-9: as 10+ but also choose a side effect:
	\begin{itemize}
	\myitem rapidly growing hair
	\myitem uncontrollable flatulence
	\myitem slurring drunk
	\myitemend tongue temporarily three feet long
	\end{itemize}
\item On a 6-: As 7-9 and you are fresh out of that special.
\end{itemize}

%% ---   ---   ---   ---   ---   
\chapter{Host}

"Little man. I have lived inside a dying star and ridden a meteorite to your planet. Put that sword away before you embarrass us both."

\marginnote{Name: Aspinal; Hammerflist; Shire; Izfr; Glockenspiel }
\marginnote{Appearance (choose 1-3): hunchbacked; oversized jaw; unearthly beauty; piercing gaze; ritual tattoos \bigskip} 
\marginnote{Statistics: \smallcaps{Fighter+2 Bard+1 }.  +1 hobby inherited from your host \bigskip}
\marginnote{Mission(Pick 1): Reverse status quo; Judge and execute; Spread the word; Protect the favoured  \bigskip} \index[stuff]{fish}
\marginnote{Demon(Pick 3) the competition; mirrors; stone; nudity; judgmental believers \bigskip} \index[stuff]{stone}  \index[stuff]{cats} 
\marginnote{Hobbies (fill in during play): 
\begin{itemize}
\item [][][][] 
\item [][][][] 
\item [][][][] 
\end{itemize}
\bigskip}
\marginnote{Experience
\begin{itemize}
\item [][][][] add a Host move
\item [][][][] +1 to any stat
\item [][][][] add a Host move
\item [][][][] +1 to any stat
\item [][][][] add a move from another playbook
\item [][][][] add a Host move
\end{itemize}}

You are an otherwordly being, angel, djinn, devil or other, possessing a mundane worshiper. Try not to get them killed, you'd need a new Host body.

\section{Third Eye} \index{Third Eye: move}
When you are challenged by someone also possessed, under magical influence or similarly strange aura \smallcaps{take +1 to wizard }rolls regarding them. If you are not sure if this triggers, ask the GM "what does my third eye see?"

\section{Fleshbound} \index{Fleshbound: move}
When you draw on more power from your patron, take 1-3 damage, \smallcaps{mark xp} then \smallcaps{roll+damage taken}.
\begin{itemize}
\item On a 10+ choose three:
	\begin{itemize}
	\myitem You sprout wings
	\myitem You can pull your favoured weapon from the air
	\myitem You impress or terrify beholders
	\myitemend You don't permanently damage your host
	\end{itemize}
\item On a 7-9 choose two
\item On a 6- the energies wrack your host warping them. Ask the Gm what this means.
\end{itemize}

\section{Fanatical fans} \index{Fanatical fans: move}
You were summoned by a cult within the dungeon. When you lead them in worship \smallcaps{take +1 bard} for that scene.
Choose why you distrust the cult that summoned you:
\begin{itemize}
\item they have an ambitious leader
\item your host was unwilling
\item they worship other things too
\item they don't respect your agency.
\end{itemize}

Choose why the cult remains important
\begin{itemize}
\item large number of followers
\item well armed and dangerous
\item master of secrete passageways and stealth
\item control a key area or choke point
\end{itemize}


\section{Belief injection} \index{Belief injection: move}
When someone swears an oath to you, even at sword point, \smallcaps{mark xp} and \smallcaps{roll+oaths received} :
\begin{itemize}
\item On a 10+ choose 2:
	\begin{itemize}
	\myitem mark the oath givers as yours visibly and permanently
	\myitem utterly destroy something with a touch
	\myitemend Sanctify the room as yours, painful to the servants of other powers.
	\end{itemize}
\item On a 7-9 choose 1
\item On a  6- your action attracts the attention of the higher powers
\end{itemize}


\section{Vene, Vidi} \index{Vene Vidi: move}
When you invade somebody's mind \smallcaps{roll+fighter} 
\begin{itemize}
\item On a 10+ you take control of the body, doing as you will with your old host. Update your health stat to match your new body.
\item On a 7-9 you take control of their body but they take control of your former host. Both of you update health as required.
\item On a 6- they throw you back out, leaving you weak and vulnerable. 
\end{itemize}


%% ---   ---   ---   ---   ---   
\chapter{Hybrid}

" He is a complex trophy of Moreau's skill, a bear, tainted with dog and ox, and one of the most elaborately made of all the creatures."

\marginnote{Name: Centaurus; It; Braggin; Platycat; Dr. Wells; M'ling }
\marginnote{Appearance (choose 1-3): richly dressed; dirty apron; ragged loincloth; bespectacled; giant wig; masked \bigskip} 
\marginnote{Statistics: \smallcaps{Thief -1 } and effects of Look in the Mirror. \bigskip}
\marginnote{Job(pick 1): Cautionary tale; Beast master; Ambassador; Barber-Surgeon \bigskip}
\marginnote{Demon(Pick 3) Masks; hunters; clockwork; perfume; lust \bigskip}
\marginnote{Hobbies (fill in during play): 
\begin{itemize}
\item [][][][] 
\item [][][][] 
\item [][][][] 
\end{itemize}
\bigskip}
\marginnote{Experience
\begin{itemize}
\item [][][][] add a Hybrid move
\item [][][][] +1 to any stat
\item [][][][] add a Hybrid move
\item [][][][] +1 to any stat
\item [][][][] add a move from another playbook
\item [][][][] add a Hybrid move
\end{itemize}}

Start with Two moves and Look in the Mirror.

\section{Look in the Mirror}
Whether an unholy experiment, the result of a curse or descendant of an ancient lineage you are a mixture of human and animal parts, perhaps multiple. Choose four from below that describe your blend species:
\begin{itemize}
\item Aggressive Predator: \smallcaps{Fighter +1}
\item Fiercely Territorial: \smallcaps{Fighter +1}
\item Long Lived: \smallcaps{Wizard +1}
\item Inherently Magical: \smallcaps{Wizard +1}
\item Lives in groups: \smallcaps{Bard +1}
\item Inherently Musical: \smallcaps{Bard +1}
\end{itemize}

\section{My Other Half} \index{My Other Half: move}
You have a reasonably loyal follower made up of the other bits\footnote{for a centaur, it would be a horses head on two human legs}. Name them, name the animal instinct that drives them and give them a single stat as per Look in the Mirror. They can use that stat to use the appropriate basic move.

\section{Crushing Empathy} \index{Crushing Empathy: move}
When you attempt to dominate, direct or control a creature that is part of your blend \smallcaps{ take +1 to rolls}

\section{Amateur Surgeon} \index{Amateur Surgeon: move}
You have a collection of books, notes and diagrams for different anatomies. When you have a patient and time to consult your notes \smallcaps{roll + wizard} 
\begin{itemize}
\item On a 10+ choose two. You may choose the same thing twice:
	\begin{itemize}
	\myitem You heal them \smallcaps{+1 health}
	\myitem You can implant an item in them
	\myitemend You can fashion an effective prosthesis.
	\end{itemize}
\item On a 7-9 choose one.
\item On a 6- choose one, but it is a temporary patch job.
\end{itemize}

\section{Forensic Zoologist} \index{Forensic Zoologist: move}
When you see a wound caused by an animal you know how serious it is, any side effects and what's the best way to approach that species.

\section{Suppressed Instincts} \index{Suppressed Instincts: move}
As long as you have been socially awkward because of your animal parts this session, you may know what any animal of that type would know when looking at something. 


%% ---   ---   ---   ---   ---   
\chapter{Elemental Ooze}

\marginnote{Name: Phibble; Puddin; Plop; Psuedopod Paul; Plum-blossom-on-the-night-breeze }
\marginnote{Appearance (choose 1-3): translucent; glowy; tentacles; fins; cloud of flies; cigar; gravelly voice \bigskip} 
\marginnote{Statistics: \smallcaps{Thief -1 } and effects of Look in the Mirror. \bigskip}
\marginnote{Job(pick 1): Cautionary tale; Beast master; Ambassador; Barber-Surgeon \bigskip}
\marginnote{Demon(Pick 3) Masks; hunters; clockwork; perfume; lust \bigskip}
\marginnote{Hobbies (fill in during play): 
\begin{itemize}
\item [][][][] 
\item [][][][] 
\item [][][][] 
\end{itemize}
\bigskip}
\marginnote{Experience
\begin{itemize}
\item [][][][] add a Hybrid move
\item [][][][] +1 to any stat
\item [][][][] add a Hybrid move
\item [][][][] +1 to any stat
\item [][][][] add a move from another playbook
\item [][][][] add a Hybrid move
\end{itemize}}

\section{Inner Core} \index{Inner Core: move}
A floating brain, skull, crystal old boot or  simialr, you you have an inner core that acts as your focus. You \smallcaps{recover five health} per long rest instead of the typical three. Anyone you allow to handle your core may \smallcaps{recover one health} immediately once per day. 

\section{Squishy} \index{Squishy: move}
When someone steps on you, you can steal their emotional energy\footnote{leaving them numb, calm or unchanged as the story suggests}. When you absorb this energy, \smallcaps{mark XP} and \smallcaps{roll + thief} 
\begin{itemize}
\item On a 10+ choose two\footnote{either choose something that fits the situation or ask the character owner why you are picking that emotion up in the background. Why is their subconscious broadcasting this?}
	\begin{itemize}
	\myitem Happiness - you can bounce like a ball
	\myitem Anger - You deal corrosive damage to all you touch
	\myitemend Fear - You go rigid and largely immune to physical harm
	\end{itemize}
\item On a 7-9 Choose one
\item On a 6- Choose one, but it only last while you display an excess of that emotion
\end{itemize}

\section{Extremeophile} \index{Extremeophlie: move}
Name an extreme environment you are completely safe and happy in - lava; underwater; buried in dry sand; pools of acid slime; tendrils of the cave anemone ect ect. What role does this environment play in the dungeon?

\section{Not from around here} \index{Not from around here: move}
When you enter a brand new area \smallcaps{roll+wizard}
\begin{itemize} 
\item On a 10+ you change the shape or form to totally suit the new environment to totally suit the new environment.
\item On a 7-9 you change shape or form but retain a critical weakness. Ask the GM what it is.
\item On a 6- you change part way but are left exhausted or vulnerable in your new form
\end{itemize}

\section{Evolution} \index{Evolution: move}
You are the epitome of your element. \footnote{Fire, Earth, Water, Air, Metal, Wood, Void sure, but what about bone, compost, gold, cheese, wasabi, paper, crystal, rust, bricks, stalactites, ash, coffee, cloth, glass, lead, cobwebs or rats? Pick something that is intrinsic to the dungeon, not to a model of the players world }. Sadly, you are not the only one. Name the element opposed to you and why you feel they are a threat.
When you consume your bodyweight of that opposed element \smallcaps{Mark XP} and \smallcaps{roll + wizard}
\begin{itemize}
\item On a 10+ You succeed in merging the two elements into something new. Say what it is and what it feels like.
\item On a 7-9 You manage to hold the two elements together for now with limited access to both, but they don't merge and you will need to redo this move after every long rest until it resolves.
\item On a 6- It goes horribly wrong. Take \smallcaps{six damage} and the GM will describe what happens to the room around you.
 \end{itemize}

%% ---   ---   ---   ---   ---   
\chapter{On Magic}

In the world of Escape the Dungeon, brave heroes explore the forgotten corners of the world, slaying creatures and collecting treasure.  They often try to kill the player characters for 'loot and experience'. How on earth does this work?

The world is steeped in magic. Magic is energy, throbbing potential for things to happen. It is unstable and tends to decay into something else quite quickly. Gold is the pure element that happens to be what a lot of magic turns into. This explains why town magic is so expensive, as the wizard is literally turning the gold back into raw magic to be infused into the spell. This process works in reverse too, and anywhere there is a large amount of magic you will have gold crystallizing out of the air. Since the dungeon largely runs on magic, you can see why heroes are attracted.

The second part of the surface dwellers' metaphysics of gold is that it can be transmuted into Levelup and back. A quantity of gold can be turned into new and terrible powers as the magic makes the target more than they were before. Traditionally, this type of personal growth up is achieved by killing, pushing things through the doorway from life to death. The bigger the door they need, the more Levelup you get. Levelup is the same thing as gold, gold is the same thing as magic, and magic changes the world.

The third part is the balance. The world has not ended in a runaway explosion of extinctions, tidal waves of gold boiling off into abstract gibbering shapes and fireballs. Magic is conserved. An increase of the amount of gold in the world must mean that the amount of free magic has decreased or that some things have died to keep the magic level the same. Or, local imbalance is corrected more directly.

In this world, large concentrations of magic create drama and gold. Large concentrations of gold leak magic, but also Levelup, normally in the form of tough, angry creatures. A rat that sleeps on a gold coin might be a giant rat next week. Sewers under rich cities are dangerous places. Owning a bank vault or being a tax collector is a scary proposition. Monsters don't carry gold, some of their life force crystallises into it when some idiot hero stabs them. Dragons sleep on piles of gold to absorb the magic they need to exist. Summoning a major spell can cause nearby gold to vanish, and if there's not enough gold, for life force to get sucked out of the wizard or someone nearby. A hero levelling up is drinking up magic that the dungeon needs to survive. Gold stolen and spent in the town is magic lost to the dungeon. A powerful monster is a significant investment of resources by the dungeon, but worth it if it stops some handsome thief crashing the local economy.















%% ---   ---   ---   ---   ---   
\chapter{Game Master Playbook}

The agenda and principles again:

\bigskip
\section{Agenda}\label{sec:Agenda}\index{agenda}
\begin{itemize}
    \item Agenda
	\begin{itemize}
	\myitem Play to find out what happens
	\myitem Make the players'  character's lives not boring
	\myitemend Subvert that which is taken from granted
	\end{itemize}
\end{itemize}
The Agenda is the key thing. If you, as GM, are in doubt what to do, choose the thing that follows the agenda. It's not a bad guide for players either.

\bigskip
\section{Principles}\label{sec:Principles}\index{principles}
\begin{itemize}
\item Principles
	\begin{itemize}
	\myitem don't waste your players' time
	\myitem the dungeon pressures to conform
	\myitem take tropes to their logical, nonsensical extreme
	\myitem sprinkle details of everyday fantasy everywhere 
	\myitem make the dungeon seem fantastically real
	\myitem name everyone, make everyone rational within their role
	\myitem build a bigger dungeon through play, not plot
	\myitem create interesting dilemmas not interesting traps
	\myitem address yourself to the characters not players
	\myitem make your move, but never speak its name
	\myitem ask loaded questions and build on the answers 
	\myitemend sometimes, reflect a question back upon the players 
	\end{itemize}
\end{itemize}


%% ---   ---   ---   ---   ---   
\chapter{Designer Notes}

Rule 1 - No Exceptions
Rule 2 - Needs to work for a one shot with strangers at a conference
Rule 3 - Evoke, not mimic, old school fighting fantasy
Rule 4 - Support the GM
Rule 5 - Every mechanical part needs to justify its existence
Rule 6 - Every move needs to relate back to theme or the mechanics
Rule 7 - No stat substitution moves

on Playbooks
a) Does this playbook have a niche. Is it overshadowed?
b) Is the playbook fun and coherent?
c) Does the playbook offer levers for the GM to pull?
d) Does the playbook offer a second, different way to play that character?

On Moves
i) Is the trigger only going to trigger when it is important?
ii) does a miss move the sotry forward with something interesting?
iii) Does a success?
iv) 

Formatting:
Only stats use digits. Hold is spent in one or twos not 1 or 2. Add explanation for hold or choice from list under the first option that uses it.
All moves to use the same format of trigger, 10+, 7-9 then 6-


%%
% The back matter contains appendices, bibliographies, indices, glossaries, etc.







\backmatter

\bibliography{sample-handout}
\bibliographystyle{plainnat}

\printindex[stuff]

\printindex

\end{document}

